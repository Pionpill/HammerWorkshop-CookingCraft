\documentclass{PionpillNote-book}

\title{\Huge{开发设计说明书} \\ \large{烹饪工艺(CookingCraft)}}
\author{
    北岸\footnote{笔名:北岸,英文笔名:Pionpill,锤子工坊负责人}
}
\date{\today}


\begin{document}


\maketitle                  % 标题

\vspace{2cm}
\noindent\textbf{前言:}

此文档为开发人员(主要为程序员)设计说明书,旨在给予模组设计统一的方向。同时,其它开发人员也可阅读此文档的部分内容,了解设计方向。

本文所有数值设计,模型设计以及逻辑关系均以基岩版为准,暂不考虑开发 Java 版。

\noindent\textbf{成员:}

北岸

\noindent\textbf{注:}

这并不是标准的程序详细设计说明书。由于我们是小规模团队,美工等其它方面的内容也会被本文中被提及。此外,本人会用到 UML 的一些关系图,虽然基岩版的开发并不会涉及到大量的高级语言源代码开发,但在脚本文件(python)和 json 文件中,UML 图像也能提供一个有效的帮助。

\hfill \textbf{版本:v1.0.0}


\newpage
\tableofcontents            % 目录
\thispagestyle{empty}
\newpage
\setcounter{page}{1}

\import{sections}{styles.tex}
\import{sections}{section-1.tex}
\import{sections}{section-2.tex}
\import{sections}{section-3.tex}
\import{sections}{section-4.tex}
\import{sections}{section-5.tex}
\import{sections}{section-6.tex}
\import{sections}{section-7.tex}
\import{sections}{section-8.tex}

% \import{sections}{appendix.tex}


\end{document}