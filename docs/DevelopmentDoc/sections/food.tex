\section{食物设计}

\subsection{简易食物}

\subsubsection{熟玉米}

熟玉米品质为良好,在多个工作台上均可合成,合成表如下:

\import{../tikz/recipes/baking_furnace}{cooked_corn.tex}
\import{../tikz/recipes/food_steamer}{cooked_corn.tex}
\import{../tikz/recipes/grill}{cooked_corn.tex}
\import{../tikz/recipes/stew_pot}{cooked_corn.tex}

类似的,其他简易食物若能烹饪,则在配方如上所示。

熟玉米数值设计:

经过加工与烘焙,可确定生成物携带的饥饿值与营养值。
\begin{equation}
    \begin{aligned}
        H_{\text{熟玉米}} & = 3 \times 0.8 + 2 = 4.5 \nonumber
    \end{aligned}
\end{equation}

\begin{table}[H]
    \centering
    \caption{熟玉米配方}
    \setlength{\tabcolsep}{4mm}
    \begin{tabular}{c|ccc|cc}
        \toprule
        \textbf{物品} & \textbf{品质} & \textbf{饥饿值} & \textbf{营养值} & \textbf{可食用} & \textbf{效果}\\
        \midrule
        熟玉米 & $\bigstar \bigstar$ & 4 & 1.2 & 是 & - \\
        \bottomrule
    \end{tabular}
\end{table}


\subsection{烘焙食物}

烘焙食物营养值设定为 1.2。烘焙食物都必须经过烘焙炉获取最终产品。

\subsubsection{苹果派}

苹果派品质为:大师,合成表如下:

\import{../tikz/recipes/cooking_table}{apple_pie.tex}

苹果派数值如下:

\begin{table}[H]
    \centering
    \caption{苹果派配方}
    \setlength{\tabcolsep}{4mm}
    \begin{tabular}{c|ccc|cc}
        \toprule
        \textbf{物品} & \textbf{品质} & \textbf{饥饿值} & \textbf{营养值} & \textbf{可食用} & \textbf{效果}\\
        \midrule
        生苹果派 & $\bigstar$ & 5 & 0.6 & 是 & 反胃 30\% 5s \\
        苹果派 & $\bigstar \bigstar \bigstar \bigstar$ & 12 & 1.2 & 是 & 生命恢复II 10s(8生命) \\
        \bottomrule
    \end{tabular}
\end{table}

\subsubsection{胡萝卜派}

胡萝卜派合成表如下:

\import{../tikz/recipes/cooking_table}{carrot_pie.tex}

胡萝卜派数值设计:

\begin{table}[H]
    \centering
    \caption{胡萝卜派配方}
    \setlength{\tabcolsep}{4mm}
    \begin{tabular}{c|ccc|cc}
        \toprule
        \textbf{物品} & \textbf{品质} & \textbf{饥饿值} & \textbf{营养值} & \textbf{可食用} & \textbf{效果}\\
        \midrule
        生胡萝卜派 & $\bigstar$ &4 & 0.6 & 是 & 反胃 30\% 5s \\
        胡萝卜派 & $\bigstar \bigstar \bigstar$ & 10 & 1.2 & 是 & 无 \\
        \bottomrule
    \end{tabular}
\end{table}

\subsubsection{鸡蛋派}

鸡蛋派合成表如下:

\import{../tikz/recipes/cooking_table}{egg_pie.tex}

鸡蛋派数值设计:

\begin{table}[H]
    \centering
    \caption{鸡蛋派配方}
    \setlength{\tabcolsep}{4mm}
    \begin{tabular}{c|ccc|cc}
        \toprule
        \textbf{物品} & \textbf{品质} & \textbf{饥饿值} & \textbf{营养值} & \textbf{可食用} & \textbf{效果}\\
        \midrule
        生鸡蛋派 & $\bigstar$ & 3 & 0.6 & 是 & 反胃 100\% 10s, 中毒1 50\% 5s \\
        鸡蛋派 & $\bigstar \bigstar \bigstar$ & 9 & 1.2 & 是 & 无 \\
        \bottomrule
    \end{tabular}
\end{table}

\subsubsection{浆果派}

浆果派合成表如下:

\import{../tikz/recipes/cooking_table}{berry_pie.tex}

浆果派数值设计:

\begin{table}[H]
    \centering
    \caption{浆果派配方}
    \setlength{\tabcolsep}{4mm}
    \begin{tabular}{c|ccc|cc}
        \toprule
        \textbf{物品} & \textbf{品质} & \textbf{饥饿值} & \textbf{营养值} & \textbf{可食用} & \textbf{效果}\\
        \midrule
        生浆果派 & $\bigstar$ & 5 & 0.6 & 是 & 反胃 50\% 8s, 中毒1 10\% 5s \\
        浆果派 & $\bigstar \bigstar \bigstar \bigstar$ & 12 & 1.2 & 是 & 夜视,2min \\
        \bottomrule
    \end{tabular}
\end{table}

\newpage