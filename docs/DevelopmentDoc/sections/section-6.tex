\section{物品设计}
\subsection{烹饪材料}

\subsubsection{面粉}

面粉,使用石磨合成。

\begin{figure}[H]
    \centering
    \begin{tikzpicture}
        \draw [recipe grid](0,0) grid (1,1);
        \draw [recipe grid](4,0) grid (6,1);
        \node (1) at (0.5,0.5) {\includegraphics[width=0.8cm,height=0.8cm]{./images/origin/wheat.png}};
        \node (2)  at (4.5,0.5) {\includegraphics[width=0.8cm,height=0.8cm]{./images/mod/flour.png}};
        \node (3)  at (5.5,0.5) {\includegraphics[width=0.8cm,height=0.8cm]{./images/mod/straw.png}};
        \begin{scope}
            \draw [-{Stealth},thick] (1.5,0.5) -- (3.5,0.5);
        \end{scope}
        \begin{scope}[xshift = 0.35cm, yshift = -0.35cm]
            \node (1sub) at (0.5,0.5) [font=\scriptsize] {3};
        \end{scope}
    \end{tikzpicture}
    \caption{面粉 UI}
\end{figure}

\begin{table}[H]
    \centering
    \caption{面粉配方}
    \setlength{\tabcolsep}{4mm}
    \begin{tabular}{c|cc|cc}
        \toprule
        \textbf{物品} & \textbf{饥饿值} & \textbf{营养值} & \textbf{可食用} & \textbf{效果}\\
        \midrule
        面粉 & 5 & 1.5 & 否 & 无 \\
        \bottomrule
    \end{tabular}
\end{table}

\subsubsection{大米}

大米,使用石磨合成。

\begin{figure}[H]
    \centering
    \begin{tikzpicture}
        \draw [recipe grid](0,0) grid (1,1);
        \draw [recipe grid](4,0) grid (6,1);
        \node (1) at (0.5,0.5) {\includegraphics[width=0.8cm,height=0.8cm]{./images/mod/rice.png}};
        \node (2)  at (4.5,0.5) {\includegraphics[width=0.8cm,height=0.8cm]{./images/mod/rice_pieces.png}};
        \node (3)  at (5.5,0.5) {\includegraphics[width=0.8cm,height=0.8cm]{./images/mod/straw.png}};
        \begin{scope}
            \draw [-{Stealth},thick] (1.5,0.5) -- (3.5,0.5);
        \end{scope}
        \begin{scope}[xshift = 0.35cm, yshift = -0.35cm]
            \node (1sub) at (0.5,0.5) [font=\scriptsize] {3};
        \end{scope}
    \end{tikzpicture}
    \caption{大米配方}
\end{figure}

\begin{table}[H]
    \centering
    \caption{大米配方}
    \setlength{\tabcolsep}{4mm}
    \begin{tabular}{c|cc|cc}
        \toprule
        \textbf{物品} & \textbf{饥饿值} & \textbf{营养值} & \textbf{可食用} & \textbf{效果}\\
        \midrule
        大米 & 5 & 1.5 & 否 & 无 \\
        \bottomrule
    \end{tabular}
\end{table}

\subsubsection{淀粉}

淀粉,使用石磨合成。

\begin{figure}[H]
    \centering
    \begin{tikzpicture}
        \draw [recipe grid](0,0) grid (1,1);
        \draw [recipe grid](4,0) grid (6,1);
        \node (1) at (0.5,0.5) {\includegraphics[width=0.8cm,height=0.8cm]{./images/mod/rice_pieces.png}};
        \node (2)  at (4.5,0.5) {\includegraphics[width=0.8cm,height=0.8cm]{./images/mod/rice_flour.png}};
        \begin{scope}
            \draw [-{Stealth},thick] (1.5,0.5) -- (3.5,0.5);
        \end{scope}
    \end{tikzpicture}
    \caption{淀粉配方}
\end{figure}

\begin{table}[H]
    \centering
    \caption{淀粉配方}
    \setlength{\tabcolsep}{4mm}
    \begin{tabular}{c|cc|cc}
        \toprule
        \textbf{物品} & \textbf{饥饿值} & \textbf{营养值} & \textbf{可食用} & \textbf{效果}\\
        \midrule
        淀粉 & 5 & 1.5 & 否 & 无 \\
        \bottomrule
    \end{tabular}
\end{table}

\subsubsection{玉米粒}

玉米粒,使用石磨合成。

\begin{figure}[H]
    \centering
    \begin{tikzpicture}
        \draw [recipe grid](0,0) grid (1,1);
        \draw [recipe grid](4,0) grid (6,1);
        \node (1) at (0.5,0.5) {\includegraphics[width=0.8cm,height=0.8cm]{./images/mod/corn.png}};
        \node (2)  at (4.5,0.5) {\includegraphics[width=0.8cm,height=0.8cm]{./images/mod/corn_pieces.png}};
        \node (3)  at (5.5,0.5) {\includegraphics[width=0.8cm,height=0.8cm]{./images/mod/straw.png}};
        \begin{scope}
            \draw [-{Stealth},thick] (1.5,0.5) -- (3.5,0.5);
        \end{scope}
        \begin{scope}[xshift = 0.35cm, yshift = -0.35cm]
            \node (1sub) at (0.5,0.5) [font=\scriptsize] {3};
        \end{scope}
    \end{tikzpicture}
    \caption{玉米粒配方}
\end{figure}

\begin{table}[H]
    \centering
    \caption{玉米粒配方}
    \setlength{\tabcolsep}{4mm}
    \begin{tabular}{c|cc|cc}
        \toprule
        \textbf{物品} & \textbf{饥饿值} & \textbf{营养值} & \textbf{可食用} & \textbf{效果}\\
        \midrule
        玉米粒 & 7 & 1.5 & 否 & 无 \\
        \bottomrule
    \end{tabular}
\end{table}

\subsubsection{玉米粉}

玉米粉,使用石磨合成。

\begin{figure}[H]
    \centering
    \begin{tikzpicture}
        \draw [recipe grid](0,0) grid (1,1);
        \draw [recipe grid](4,0) grid (6,1);
        \node (1) at (0.5,0.5) {\includegraphics[width=0.8cm,height=0.8cm]{./images/mod/corn_pieces.png}};
        \node (2)  at (4.5,0.5) {\includegraphics[width=0.8cm,height=0.8cm]{./images/mod/corn_flour.png}};
        \begin{scope}
            \draw [-{Stealth},thick] (1.5,0.5) -- (3.5,0.5);
        \end{scope}
    \end{tikzpicture}
    \caption{玉米粉配方}
\end{figure}

\begin{table}[H]
    \centering
    \caption{玉米粉配方}
    \setlength{\tabcolsep}{4mm}
    \begin{tabular}{c|cc|cc}
        \toprule
        \textbf{物品} & \textbf{饥饿值} & \textbf{营养值} & \textbf{可食用} & \textbf{效果}\\
        \midrule
        玉米粉 & 7 & 1.5 & 否 & 无 \\
        \bottomrule
    \end{tabular}
\end{table}


\subsubsection{香料}

香料,使用石磨合成。

\begin{figure}[H]
    \centering
    \begin{tikzpicture}
        \draw [recipe grid](0,0) grid (1,1);
        \draw [recipe grid](4,0) grid (6,1);
        \node (1) at (0.5,0.5) {\includegraphics[width=0.8cm,height=0.8cm]{./images/mod/herb.png}};
        \node (2)  at (4.5,0.5) {\includegraphics[width=0.8cm,height=0.8cm]{./images/mod/spices.png}};
        \node (3)  at (5.5,0.5) {\includegraphics[width=0.8cm,height=0.8cm]{./images/mod/straw.png}};
        \begin{scope}
            \draw [-{Stealth},thick] (1.5,0.5) -- (3.5,0.5);
        \end{scope}
        \begin{scope}[xshift = 0.35cm, yshift = -0.35cm]
            \node (1sub) at (0.5,0.5) [font=\scriptsize] {3};
        \end{scope}
    \end{tikzpicture}
    \caption{香料配方}
\end{figure}

\begin{table}[H]
    \centering
    \caption{香料配方}
    \setlength{\tabcolsep}{4mm}
    \begin{tabular}{c|cc|cc}
        \toprule
        \textbf{物品} & \textbf{饥饿值} & \textbf{营养值} & \textbf{可食用} & \textbf{效果}\\
        \midrule
        香料 & - & 1.5 & 否 & 无 \\
        \bottomrule
    \end{tabular}
\end{table}

\subsubsection{辣椒粉}

辣椒粉(chill\_powder),使用石磨合成。

\begin{figure}[H]
    \centering
    \begin{tikzpicture}
        \draw [recipe grid](0,0) grid (1,1);
        \draw [recipe grid](4,0) grid (6,1);
        \node (1) at (0.5,0.5) {\includegraphics[width=0.8cm,height=0.8cm]{./images/mod/pepper.png}};
        \node (2)  at (4.5,0.5) {\includegraphics[width=0.8cm,height=0.8cm]{./images/mod/chill_powder.png}};
        \node (3)  at (5.5,0.5) {\includegraphics[width=0.8cm,height=0.8cm]{./images/mod/straw.png}};
        \begin{scope}
            \draw [-{Stealth},thick] (1.5,0.5) -- (3.5,0.5);
        \end{scope}
    \end{tikzpicture}
    \caption{辣椒粉配方}
\end{figure}

\begin{table}[H]
    \centering
    \caption{辣椒粉配方}
    \setlength{\tabcolsep}{4mm}
    \begin{tabular}{c|cc|cc}
        \toprule
        \textbf{物品} & \textbf{饥饿值} & \textbf{营养值} & \textbf{可食用} & \textbf{效果}\\
        \midrule
        辣椒粉 & - & 1.5 & 否 & 无 \\
        \bottomrule
    \end{tabular}
\end{table}

\subsubsection{调料}

调料,使用厨务台合成。

\begin{figure}[H]
    \centering
    \begin{tikzpicture}
        \draw [recipe grid](0,0) grid (3,3);
        \begin{scope}
            \node (1) at (0.5,2.5) {};
            \node (2) at (1.5,2.5) {};
            \node (3) at (2.5,2.5) {};
            \node (4) at (0.5,1.5) {\includegraphics[width=0.8cm,height=0.8cm]{./images/mod/spices.png}};
            \node (5) at (1.5,1.5) {\includegraphics[width=0.8cm,height=0.8cm]{./images/mod/chill_powder.png}};
            \node (6) at (2.5,1.5) {};
            \node (7) at (0.5,0.5) {};
            \node (8) at (1.5,0.5) {};
            \node (9) at (2.5,0.5) {};
            \node (10) at (6.5,1.5) {\includegraphics[width=0.8cm,height=0.8cm]{./images/mod/seasoning.png}};
        \end{scope}
        \begin{scope}[xshift = 0.35cm, yshift = -0.35cm]
            \node (1sub) at (0.5,1.5) [font=\scriptsize] {2};
            \node (2sub)  at (6.5,1.5) [font=\scriptsize] {3};
        \end{scope}
        \begin{scope}
            \draw [-{Stealth},thick] (3.5,1.5) -- (5.5,1.5);
            \draw [recipe grid] (6,1) grid (7,2);
            \node (text) [font=\small] at (4.5,1.2) {厨务台};
        \end{scope}
    \end{tikzpicture}
    \caption{调料}
\end{figure}

\begin{table}[H]
    \centering
    \caption{调料配方}
    \setlength{\tabcolsep}{4mm}
    \begin{tabular}{c|cc|cc}
        \toprule
        \textbf{物品} & \textbf{饥饿值} & \textbf{营养值} & \textbf{可食用} & \textbf{效果}\\
        \midrule
        调料 & - & 1.5 & 否 & 无 \\
        \bottomrule
    \end{tabular}
\end{table}

\subsection{烘焙食物}

烘焙食物营养值设定为 1.2。烘焙食物都必须经过烘焙炉获取最终产品。

\subsubsection{烤玉米}

烤玉米品质为良好,合成表如下:

\begin{figure}[H]
    \centering
    \begin{tikzpicture}
        \begin{scope}
            \node (1) at (6.5,1.5) {\includegraphics[width=0.8cm,height=0.8cm]{./images/mod/corn.png}};
            \node (2) at (10.5,1.5) {\includegraphics[width=0.8cm,height=0.8cm]{./images/mod/baking_corn.png}};
        \end{scope}
        \begin{scope}[xshift = 0.35cm, yshift = -0.35cm]
            \node (1sub)  at (6.5,1.5) [font=\scriptsize] {};
            \node (2sub)  at (10.5,1.5) [font=\scriptsize] {};
        \end{scope}
        \begin{scope}
            \draw [-{Stealth},thick] (7.5,1.5) -- (9.5,1.5);
            \draw [recipe grid] (6,1) grid (7,2);
            \draw [recipe grid] (10,1) grid (11,2);
            \node (text) [font=\small] at (8.5,1.2) {烘焙炉};
        \end{scope}
    \end{tikzpicture}
    \caption{烤玉米}
\end{figure}

烤玉米数值设计:

经过加工与烘焙,可确定生成物携带的饥饿值与营养值。
\begin{equation}
    \begin{aligned}
        H_{\text{烤玉米}} & = 3 \times 0.8 + 2 = 4.5 \nonumber
    \end{aligned}
\end{equation}

\begin{table}[H]
    \centering
    \caption{烤玉米配方}
    \setlength{\tabcolsep}{4mm}
    \begin{tabular}{c|ccc|cc}
        \toprule
        \textbf{物品} & \textbf{品质} & \textbf{饥饿值} & \textbf{营养值} & \textbf{可食用} & \textbf{效果}\\
        \midrule
        烤玉米 & $\bigstar \bigstar$ & 4 & 1.2 & 是 & - \\
        \bottomrule
    \end{tabular}
\end{table}

\subsubsection{苹果派}

苹果派品质为:精致,合成表如下:

\begin{figure}[H]
    \centering
    \begin{tikzpicture}
        \draw [recipe grid](0,0) grid (3,3);
        \begin{scope}
            \node (1) at (0.5,2.5) {};
            \node (2) at (1.5,2.5) {\includegraphics[width=0.8cm,height=0.8cm]{./images/mod/seasoning.png}};
            \node (3) at (2.5,2.5) {};
            \node (4) at (0.5,1.5) {\includegraphics[width=0.8cm,height=0.8cm]{./images/origin/apple.png}};
            \node (5) at (1.5,1.5) {\includegraphics[width=0.8cm,height=0.8cm]{./images/origin/sugar.png}};
            \node (6) at (2.5,1.5) {\includegraphics[width=0.8cm,height=0.8cm]{./images/origin/apple.png}};
            \node (7) at (0.5,0.5) {\includegraphics[width=0.8cm,height=0.8cm]{./images/mod/flour.png}};
            \node (8) at (1.5,0.5) {\includegraphics[width=0.8cm,height=0.8cm]{./images/mod/flour.png}};
            \node (9) at (2.5,0.5) {\includegraphics[width=0.8cm,height=0.8cm]{./images/mod/flour.png}};
            \node (10) at (6.5,1.5) {\includegraphics[width=0.8cm,height=0.8cm]{./images/mod/raw_apple_pie.png}};
            \node (11) at (10.5,1.5) {\includegraphics[width=0.8cm,height=0.8cm]{./images/mod/apple_pie.png}};
        \end{scope}
        \begin{scope}[xshift = 0.35cm, yshift = -0.35cm]
            \node (5sub) at (1.5,1.5) [font=\scriptsize] {3};
            \node (10sub)  at (6.5,1.5) [font=\scriptsize] {3};
            \node (11sub)  at (10.5,1.5) [font=\scriptsize] {3};
        \end{scope}
        \begin{scope}
            \draw [-{Stealth},thick] (3.5,1.5) -- (5.5,1.5);
            \draw [-{Stealth},thick] (7.5,1.5) -- (9.5,1.5);
            \draw [recipe grid] (6,1) grid (7,2);
            \draw [recipe grid] (10,1) grid (11,2);
            \node (text) [font=\small] at (4.5,1.2) {厨务台};
            \node (text) [font=\small] at (8.5,1.2) {烘焙炉};
        \end{scope}
    \end{tikzpicture}
    \caption{苹果派}
\end{figure}

苹果派数值设计:

由下列公式计算得出平均每个苹果派原材料携带饥饿值:
\begin{equation}
    \frac{6 \times 3 + 4 \times 2 + 1.5 \times 3}{3} = 10 \nonumber
\end{equation}

经过加工与烘焙,可确定生成物携带的饥饿值与营养值。
\begin{equation}
    \begin{aligned}
        H_{\text{生苹果派}} & = 10 \times 0.4 = 4 \\
        H_{\text{苹果派}} & = 10 \times 0.8 + 2 = 10 \nonumber
    \end{aligned}
\end{equation}

\begin{table}[H]
    \centering
    \caption{苹果派配方}
    \setlength{\tabcolsep}{4mm}
    \begin{tabular}{c|ccc|cc}
        \toprule
        \textbf{物品} & \textbf{品质} & \textbf{饥饿值} & \textbf{营养值} & \textbf{可食用} & \textbf{效果}\\
        \midrule
        生苹果派 & $\bigstar$ &4 & 0.6 & 是 & 反胃 30\% 5s \\
        苹果派 & $\bigstar \bigstar \bigstar$ &10 & 1.2 & 是 & 生命恢复1 5s \\
        \bottomrule
    \end{tabular}
\end{table}

\subsubsection{胡萝卜派}

胡萝卜派合成表如下:

\begin{figure}[H]
    \centering
    \begin{tikzpicture}
        \draw [recipe grid](0,0) grid (3,3);
        \begin{scope}
            \node (1) at (0.5,2.5) {};
            \node (2) at (1.5,2.5) {\includegraphics[width=0.8cm,height=0.8cm]{./images/origin/sugar.png}};
            \node (3) at (2.5,2.5) {};
            \node (4) at (0.5,1.5) {\includegraphics[width=0.8cm,height=0.8cm]{./images/origin/carrot.png}};
            \node (5) at (1.5,1.5) {\includegraphics[width=0.8cm,height=0.8cm]{./images/origin/carrot.png}};
            \node (6) at (2.5,1.5) {\includegraphics[width=0.8cm,height=0.8cm]{./images/origin/carrot.png}};
            \node (7) at (0.5,0.5) {\includegraphics[width=0.8cm,height=0.8cm]{./images/mod/flour.png}};
            \node (8) at (1.5,0.5) {\includegraphics[width=0.8cm,height=0.8cm]{./images/mod/flour.png}};
            \node (9) at (2.5,0.5) {\includegraphics[width=0.8cm,height=0.8cm]{./images/mod/flour.png}};
            \node (10) at (6.5,1.5) {\includegraphics[width=0.8cm,height=0.8cm]{./images/mod/raw_carrot_pie.png}};
            \node (11) at (10.5,1.5) {\includegraphics[width=0.8cm,height=0.8cm]{./images/mod/carrot_pie.png}};
        \end{scope}
        \begin{scope}[xshift = 0.35cm, yshift = -0.35cm]
            \node (2sub) at (1.5,2.5) [font=\scriptsize] {3};
            \node (10sub)  at (6.5,1.5) [font=\scriptsize] {3};
            \node (11sub)  at (10.5,1.5) [font=\scriptsize] {3};
        \end{scope}
        \begin{scope}
            \draw [-{Stealth},thick] (3.5,1.5) -- (5.5,1.5);
            \draw [-{Stealth},thick] (7.5,1.5) -- (9.5,1.5);
            \draw [recipe grid] (6,1) grid (7,2);
            \draw [recipe grid] (10,1) grid (11,2);
            \node (text) [font=\small] at (8.5,1.8) {3pf};
            \node (text) [font=\small] at (4.5,1.2) {厨务台};
            \node (text) [font=\small] at (8.5,1.2) {烘焙炉};
        \end{scope}
    \end{tikzpicture}
    \caption{胡萝卜派}
\end{figure}

胡萝卜派数值设计:

由下列公式计算得出平均每个胡萝卜派原材料携带饥饿值:
\begin{equation}
    \frac{6 \times 3 + 3 \times 3 + 1.5 \times 3}{3} = 10.5 \nonumber
\end{equation}

经过加工与烘焙,可确定生成物携带的饥饿值与营养值。
\begin{equation}
    \begin{aligned}
        H_{\text{生胡萝卜派}} & = \lfloor 10.5 \times 0.4 \rfloor = 4 \\
        H_{\text{胡萝卜派}} & = \lfloor 10.5 \times 0.8 + 3 \rfloor = 11 \nonumber
    \end{aligned}
\end{equation}

\begin{table}[H]
    \centering
    \caption{胡萝卜派配方}
    \setlength{\tabcolsep}{4mm}
    \begin{tabular}{c|ccc|cc}
        \toprule
        \textbf{物品} & \textbf{品质} & \textbf{饥饿值} & \textbf{营养值} & \textbf{可食用} & \textbf{效果}\\
        \midrule
        生胡萝卜派 & $\bigstar$ &4 & 0.6 & 是 & 反胃 30\% 5s \\
        胡萝卜派 & $\bigstar \bigstar \bigstar$ & 11 & 1.2 & 是 & 无 \\
        \bottomrule
    \end{tabular}
\end{table}

\subsubsection{鸡蛋派}

鸡蛋派合成表如下:

\begin{figure}[H]
    \centering
    \begin{tikzpicture}
        \draw [recipe grid](0,0) grid (3,3);
        \begin{scope}
            \node (1) at (0.5,2.5) {\includegraphics[width=0.8cm,height=0.8cm]{./images/mod/seasoning.png}};
            \node (2) at (1.5,2.5) {\includegraphics[width=0.8cm,height=0.8cm]{./images/origin/sugar.png}};
            \node (3) at (2.5,2.5) {\includegraphics[width=0.8cm,height=0.8cm]{./images/mod/seasoning.png}};
            \node (4) at (0.5,1.5) {\includegraphics[width=0.8cm,height=0.8cm]{./images/origin/egg.png}};
            \node (5) at (1.5,1.5) {\includegraphics[width=0.8cm,height=0.8cm]{./images/origin/egg.png}};
            \node (6) at (2.5,1.5) {\includegraphics[width=0.8cm,height=0.8cm]{./images/origin/egg.png}};
            \node (7) at (0.5,0.5) {\includegraphics[width=0.8cm,height=0.8cm]{./images/mod/flour.png}};
            \node (8) at (1.5,0.5) {\includegraphics[width=0.8cm,height=0.8cm]{./images/mod/flour.png}};
            \node (9) at (2.5,0.5) {\includegraphics[width=0.8cm,height=0.8cm]{./images/mod/flour.png}};
            \node (10) at (6.5,1.5) {\includegraphics[width=0.8cm,height=0.8cm]{./images/mod/raw_egg_pie.png}};
            \node (11) at (10.5,1.5) {\includegraphics[width=0.8cm,height=0.8cm]{./images/mod/egg_pie.png}};
        \end{scope}
        \begin{scope}[xshift = 0.35cm, yshift = -0.35cm]
            \node (2sub) at (1.5,2.5) [font=\scriptsize] {3};
            \node (10sub)  at (6.5,1.5) [font=\scriptsize] {3};
            \node (11sub)  at (10.5,1.5) [font=\scriptsize] {3};
        \end{scope}
        \begin{scope}
            \draw [-{Stealth},thick] (3.5,1.5) -- (5.5,1.5);
            \draw [-{Stealth},thick] (7.5,1.5) -- (9.5,1.5);
            \draw [recipe grid] (6,1) grid (7,2);
            \draw [recipe grid] (10,1) grid (11,2);
            \node (text) [font=\small] at (8.5,1.8) {3pf};
            \node (text) [font=\small] at (4.5,1.2) {厨务台};
            \node (text) [font=\small] at (8.5,1.2) {烘焙炉};
        \end{scope}
    \end{tikzpicture}
    \caption{鸡蛋派}
\end{figure}

鸡蛋派数值设计:

由下列公式计算得出平均每个鸡蛋派原材料携带饥饿值:
\begin{equation}
    \frac{6 \times 3 + 2 \times 3 + 1.5 \times 3}{3} = 9.5 \nonumber
\end{equation}

经过加工与烘焙,可确定生成物携带的饥饿值与营养值。
\begin{equation}
    \begin{aligned}
        H_{\text{生鸡蛋派}} & = \lfloor 9.5 \times 0.4 \rfloor = 3 \\
        H_{\text{鸡蛋派}} & = \lfloor 9.5 \times 0.8 + 2 \rfloor = 9 \nonumber
    \end{aligned}
\end{equation}

\begin{table}[H]
    \centering
    \caption{鸡蛋派配方}
    \setlength{\tabcolsep}{4mm}
    \begin{tabular}{c|ccc|cc}
        \toprule
        \textbf{物品} & \textbf{品质} & \textbf{饥饿值} & \textbf{营养值} & \textbf{可食用} & \textbf{效果}\\
        \midrule
        生鸡蛋派 & $\bigstar$ & 3 & 0.6 & 是 & 反胃 100\% 10s, 中毒1 50\% 5s \\
        鸡蛋派 & $\bigstar \bigstar \bigstar$ & 9 & 1.2 & 是 & 无 \\
        \bottomrule
    \end{tabular}
\end{table}

\subsubsection{浆果派}

浆果派合成表如下:

\begin{figure}[H]
    \centering
    \begin{tikzpicture}
        \draw [recipe grid](0,0) grid (3,3);
        \begin{scope}
            \node (1) at (0.5,2.5) {};
            \node (2) at (1.5,2.5) {};
            \node (3) at (2.5,2.5) {};
            \node (4) at (0.5,1.5) {\includegraphics[width=0.8cm,height=0.8cm]{./images/origin/sweet_berries.png}};
            \node (5) at (1.5,1.5) {\includegraphics[width=0.8cm,height=0.8cm]{./images/origin/glow_berries.png}};
            \node (6) at (2.5,1.5) {\includegraphics[width=0.8cm,height=0.8cm]{./images/origin/sweet_berries.png}};
            \node (7) at (0.5,0.5) {\includegraphics[width=0.8cm,height=0.8cm]{./images/mod/flour.png}};
            \node (8) at (1.5,0.5) {\includegraphics[width=0.8cm,height=0.8cm]{./images/mod/flour.png}};
            \node (9) at (2.5,0.5) {\includegraphics[width=0.8cm,height=0.8cm]{./images/mod/flour.png}};
            \node (10) at (6.5,1.5) {\includegraphics[width=0.8cm,height=0.8cm]{./images/mod/raw_berry_pie.png}};
            \node (11) at (10.5,1.5) {\includegraphics[width=0.8cm,height=0.8cm]{./images/mod/berry_pie.png}};
        \end{scope}
        \begin{scope}[xshift = 0.35cm, yshift = -0.35cm]
            \node (2sub) at (1.5,1.5) [font=\scriptsize] {3};
            \node (10sub)  at (6.5,1.5) [font=\scriptsize] {3};
            \node (11sub)  at (10.5,1.5) [font=\scriptsize] {3};
        \end{scope}
        \begin{scope}
            \draw [-{Stealth},thick] (3.5,1.5) -- (5.5,1.5);
            \draw [-{Stealth},thick] (7.5,1.5) -- (9.5,1.5);
            \draw [recipe grid] (6,1) grid (7,2);
            \draw [recipe grid] (10,1) grid (11,2);
            \node (text) [font=\small] at (8.5,1.8) {3pf};
            \node (text) [font=\small] at (4.5,1.2) {厨务台};
            \node (text) [font=\small] at (8.5,1.2) {烘焙炉};
        \end{scope}
    \end{tikzpicture}
    \caption{浆果派}
\end{figure}

浆果派数值设计:

由下列公式计算得出平均每个浆果派原材料携带饥饿值:
\begin{equation}
    \frac{6 \times 3 + 2 \times 5}{3} = 9.3 \nonumber
\end{equation}

经过加工与烘焙,可确定生成物携带的饥饿值与营养值。
\begin{equation}
    \begin{aligned}
        H_{\text{生浆果派}} & = \lfloor 9.3 \times 0.4 \rfloor = 3 \\
        H_{\text{浆果派}} & = \lfloor 9.3 \times 0.8 + 2 \rfloor = 9 \nonumber
    \end{aligned}
\end{equation}

\begin{table}[H]
    \centering
    \caption{浆果派配方}
    \setlength{\tabcolsep}{4mm}
    \begin{tabular}{c|ccc|cc}
        \toprule
        \textbf{物品} & \textbf{品质} & \textbf{饥饿值} & \textbf{营养值} & \textbf{可食用} & \textbf{效果}\\
        \midrule
        生浆果派 & $\bigstar$ & 3 & 0.6 & 是 & 反胃 50\% 8s, 中毒1 10\% 5s \\
        浆果派 & $\bigstar \bigstar \bigstar$ & 9 & 1.2 & 是 & 夜视,30s \\
        \bottomrule
    \end{tabular}
\end{table}

\newpage