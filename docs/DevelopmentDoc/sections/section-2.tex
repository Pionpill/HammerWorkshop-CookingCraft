\section{总体设计}

\subsection{设计结构}

在现版本中,设计结构十分简单。即通过小麦合成面粉,再将面粉与其它食物在烘焙台上合成其他食物。

\begin{figure}[H]
    \centering
    \begin{tikzpicture}[recipe]
        \node (1) [item] at (0,0) {小麦};
        \node (2) [block] at (3,0) {合成台};
        \node (3) [item] at (6,0) {面粉};
        \node (4) [block] at (9,0) {厨务台};
        \node (5) [block] at (12,0) {烘焙炉};
        \node (6) [item] at (15,0) {烘焙食物};
        \draw (1) -- (2);
        \draw (2) -- (3);
        \draw (3) -- (4);
        \draw (4) -- (5);
        \draw (5) -- (6);
    \end{tikzpicture}
    \caption{烘焙步骤}
    % \label{fig:烘焙步骤}
\end{figure}

在后续版本中,会加入新的植物,如香草,辣椒等等等。

\subsection{版本内容}
\subsubsection{版本更新计划}

模组计划每 3 个版本一次大更新,并确定对应的版本更新内容。在测试版(v0.1 - v0.3)开发过程中,预计实现以下主要内容:
\begin{itemize}
    \item \textbf{v0.1}: 实现工作台底层脚本构建,实装厨务台与烘焙炉两个工作台,并实现派类食物的制作。
    \item \textbf{v0.2}: 增加新工作台:研磨机,试验几个自定义植被。实现烧烤架与烧烤类食物的制作,发布 beta 测试版本。
    \item \textbf{v0.3}: 完善 beta 测试版本,新增自定义动物(肉类来源),发布 gamma 测试版本。
    \item \textbf{v1.0}: 在不改变机制的情况下,大量增加游戏内容(食物,植被,动物,工作台),发布收费正式版。
\end{itemize}

\subsubsection{版本内容}

\begin{table}[H]
    \centering
    \caption{block 方块}
    \setlength{\tabcolsep}{4mm}
    \begin{tabular}{c|cc|cc}
        \toprule
        \textbf{方块名} & 英文           & \textbf{方块类型} & \textbf{说明}              & \textbf{实装版本} \\
        \midrule
        厨务台          & cooking\_table & 合成台            & 大部分食物加工处           & v0.1              \\
        烘焙炉          & baking\_furnace   & 熔炉              & 制作烘焙食物               & v0.1              \\
        研磨机          & mill       & 熔炉            & 对原材料进行研磨,需要时间 & v0.2              \\
        烧烤架          & grill          & 响应方块      & 制作烧烤                   & v0.2              \\
        压榨机          & squeezer       & 熔炉              & 压制奶酪,油               & v0.2              \\
        锅              & pot            & 熔炉              & 制作汤类食物               & v1.0              \\
        炸锅            & fryer          & 熔炉              & 制作油炸类食物             & v1.0              \\
        \midrule
        盐块            & salt\_block     & 矿物             & 熔炼为盐                   & v1.0              \\
        \bottomrule
    \end{tabular}
\end{table}

\begin{table}[H]
    \centering
    \caption{食物与相关材料}
    \setlength{\tabcolsep}{4mm}
    \begin{tabular}{c|cccc}
        \toprule
        \textbf{物品名} & 英文              & \textbf{物品类型} & \textbf{说明/食物品质}                & \textbf{实装版本} \\
        \midrule
        面粉            & flour             & 材料            & 材料   & v0.1              \\
        香料            & condiment         & 材料            & 材料  & v0.2    \\
        辣椒            & pepper            & 材料            & 材料  & v0.2    \\
        洋葱            & onion             & 材料            & 材料  & v0.2    \\
        黄油            & butter            & 材料            & 材料  & v1.0    \\
        \midrule
        番茄            & tomato            & 水果            & $\bigstar$ & v0.2 \\
        \midrule
        生派            &                   & 生食              & $\bigstar$                            & v0.1              \\
        苹果派          & apple\_pie        & 派类食物           & $\bigstar \bigstar \bigstar$         & v0.1              \\
        浆果派          & berry\_pie        & 派类食物           & $\bigstar \bigstar \bigstar$         & v0.1              \\
        胡萝卜派        & carrot\_pie       & 派类食物           & $\bigstar \bigstar \bigstar$         & v0.1              \\
        鸡蛋饼          & egg\_pie       & 派类食物           & $\bigstar \bigstar \bigstar$         & v0.1              \\
        \midrule
        奶酪            & cheese            & 食物/材料         & $\bigstar \bigstar$                  & v1.0               \\
        南瓜饼干        & pumpkin\_fritters & 食物              & $\bigstar \bigstar$                   & v1.0              \\
        煎蛋            & fired\_eggs       & 食物              & $\bigstar \bigstar$                & v1.0              \\
        \midrule
        披萨            & pizza             & 食物              & $\bigstar \bigstar \bigstar \bigstar$ & v1.0              \\
        \bottomrule
    \end{tabular}
\end{table}

\begin{table}[H]
    \centering
    \caption{植物}
    \setlength{\tabcolsep}{4mm}
    \begin{tabular}{c|ccc|cc}
        \toprule
        \textbf{植物名} & \textbf{英文} & \textbf{植物类型} & \textbf{土地} & \textbf{备注} & \textbf{实装版本} \\
        \midrule
        番茄            & tomato        & 食物              & 泥土          & 需要攀藤      & 未定              \\
        辣椒            & pepper        & 食物/材料         & 泥土          & 直接种植      & v0.2              \\
        香草            & vanilla       & 材料              & 泥土          & 直接种植      & v0.2              \\
        洋葱            & onion         & 材料              & 泥土          & 直接种植      & v0.2              \\
        \bottomrule
    \end{tabular}
\end{table}

\begin{table}[H]
    \centering
    \caption{动物}
    \setlength{\tabcolsep}{4mm}
    \begin{tabular}{c|ccc|cc}
        \toprule
        \textbf{动物名} & \textbf{英文} & \textbf{攻击类型} & \textbf{生态} & \textbf{备注} & \textbf{实装版本} \\
        \midrule
        鹿              & deer          & 友好              & 森林,草原    & 被打后逃跑    & v0.3              \\
        鹅              & goose         & 友好              & 森林,草原    & 被打后逃跑    & v0.3              \\
        核牛            & nucleus\_cow  & 敌对              & 森林,草原    & 攻击带毒      & v0.3              \\
        \bottomrule
    \end{tabular}
\end{table}

\newpage