\section{材料设计}

\subsection{厨务台}

\subsubsection{调料}

调料,使用厨务台合成。

\begin{figure}[H]
    \centering
    \begin{tikzpicture}
        \draw [recipe grid](0,0) grid (3,3);
        \begin{scope}
            \node (1) at (0.5,2.5) {};
            \node (2) at (1.5,2.5) {};
            \node (3) at (2.5,2.5) {};
            \node (4) at (0.5,1.5) {\includegraphics[width=0.8cm,height=0.8cm]{./images/mod/spices.png}};
            \node (5) at (1.5,1.5) {\includegraphics[width=0.8cm,height=0.8cm]{./images/mod/chill_powder.png}};
            \node (6) at (2.5,1.5) {};
            \node (7) at (0.5,0.5) {};
            \node (8) at (1.5,0.5) {};
            \node (9) at (2.5,0.5) {};
            \node (10) at (6.5,1.5) {\includegraphics[width=0.8cm,height=0.8cm]{./images/mod/seasoning.png}};
        \end{scope}
        \begin{scope}[xshift = 0.35cm, yshift = -0.35cm]
            \node (1sub) at (0.5,1.5) [font=\scriptsize] {2};
            \node (2sub)  at (6.5,1.5) [font=\scriptsize] {3};
        \end{scope}
        \begin{scope}
            \draw [-{Stealth},thick] (3.5,1.5) -- (5.5,1.5);
            \draw [recipe grid] (6,1) grid (7,2);
            \node (text) [font=\small] at (4.5,1.2) {厨务台};
        \end{scope}
    \end{tikzpicture}
    \caption{调料}
\end{figure}

\begin{table}[H]
    \centering
    \caption{调料配方}
    \setlength{\tabcolsep}{4mm}
    \begin{tabular}{c|cc|cc}
        \toprule
        \textbf{物品} & \textbf{饥饿值} & \textbf{营养值} & \textbf{可食用} & \textbf{效果}\\
        \midrule
        调料 & - & 1.5 & 否 & 无 \\
        \bottomrule
    \end{tabular}
\end{table}

\subsubsection{牛奶}

牛奶,使用厨务台合成。

\begin{figure}[H]
    \centering
    \begin{tikzpicture}
        \draw [recipe grid](0,0) grid (3,3);
        \begin{scope}
            \node (1) at (0.5,2.5) {};
            \node (2) at (1.5,2.5) {};
            \node (3) at (2.5,2.5) {};
            \node (4) at (0.5,1.5) {};
            \node (5) at (1.5,1.5) {\includegraphics[width=0.8cm,height=0.8cm]{./images/origin/milk_bucket.png}};
            \node (6) at (2.5,1.5) {};
            \node (7) at (0.5,0.5) {};
            \node (8) at (1.5,0.5) {};
            \node (9) at (2.5,0.5) {};
            \node (10) at (6.5,1.5) {\includegraphics[width=0.8cm,height=0.8cm]{./images/mod/milk.png}};
        \end{scope}
        \begin{scope}[xshift = 0.35cm, yshift = -0.35cm]
            \node (1sub) at (0.5,1.5) [font=\scriptsize] {};
            \node (2sub)  at (6.5,1.5) [font=\scriptsize] {3};
        \end{scope}
        \begin{scope}
            \draw [-{Stealth},thick] (3.5,1.5) -- (5.5,1.5);
            \draw [recipe grid] (6,1) grid (7,2);
            \node (text) [font=\small] at (4.5,1.2) {厨务台};
        \end{scope}
    \end{tikzpicture}
    \caption{牛奶}
\end{figure}

\begin{table}[H]
    \centering
    \caption{牛奶配方}
    \setlength{\tabcolsep}{4mm}
    \begin{tabular}{c|cc|cc}
        \toprule
        \textbf{物品} & \textbf{饥饿值} & \textbf{营养值} & \textbf{可食用} & \textbf{效果}\\
        \midrule
        牛奶 & 1 & 1.2 & 是 & 无 \\
        \bottomrule
    \end{tabular}
\end{table}

\subsubsection{蜂蜜}

蜂蜜,使用厨务台合成。

\begin{figure}[H]
    \centering
    \begin{tikzpicture}
        \draw [recipe grid](0,0) grid (3,3);
        \begin{scope}
            \node (1) at (0.5,2.5) {};
            \node (2) at (1.5,2.5) {};
            \node (3) at (2.5,2.5) {};
            \node (4) at (0.5,1.5) {};
            \node (5) at (1.5,1.5) {\includegraphics[width=0.8cm,height=0.8cm]{./images/origin/honey_bottle.png}};
            \node (6) at (2.5,1.5) {};
            \node (7) at (0.5,0.5) {};
            \node (8) at (1.5,0.5) {};
            \node (9) at (2.5,0.5) {};
            \node (10) at (6.5,1.5) {\includegraphics[width=0.8cm,height=0.8cm]{./images/mod/honey.png}};
        \end{scope}
        \begin{scope}[xshift = 0.35cm, yshift = -0.35cm]
            \node (1sub) at (0.5,1.5) [font=\scriptsize] {};
            \node (2sub)  at (6.5,1.5) [font=\scriptsize] {};
        \end{scope}
        \begin{scope}
            \draw [-{Stealth},thick] (3.5,1.5) -- (5.5,1.5);
            \draw [recipe grid] (6,1) grid (7,2);
            \node (text) [font=\small] at (4.5,1.2) {厨务台};
        \end{scope}
    \end{tikzpicture}
    \caption{蜂蜜}
\end{figure}

\begin{figure}[H]
    \centering
    \begin{tikzpicture}
        \draw [recipe grid](0,0) grid (3,3);
        \begin{scope}
            \node (1) at (0.5,2.5) {};
            \node (2) at (1.5,2.5) {};
            \node (3) at (2.5,2.5) {};
            \node (4) at (0.5,1.5) {};
            \node (5) at (1.5,1.5) {\includegraphics[width=0.8cm,height=0.8cm]{./images/origin/honey_block.png}};
            \node (6) at (2.5,1.5) {};
            \node (7) at (0.5,0.5) {};
            \node (8) at (1.5,0.5) {};
            \node (9) at (2.5,0.5) {};
            \node (10) at (6.5,1.5) {\includegraphics[width=0.8cm,height=0.8cm]{./images/mod/honey.png}};
        \end{scope}
        \begin{scope}[xshift = 0.35cm, yshift = -0.35cm]
            \node (1sub) at (0.5,1.5) [font=\scriptsize] {};
            \node (2sub)  at (6.5,1.5) [font=\scriptsize] {4};
        \end{scope}
        \begin{scope}
            \draw [-{Stealth},thick] (3.5,1.5) -- (5.5,1.5);
            \draw [recipe grid] (6,1) grid (7,2);
            \node (text) [font=\small] at (4.5,1.2) {厨务台};
        \end{scope}
    \end{tikzpicture}
    \caption{蜂蜜}
\end{figure}

\begin{table}[H]
    \centering
    \caption{蜂蜜配方}
    \setlength{\tabcolsep}{4mm}
    \begin{tabular}{c|cc|cc}
        \toprule
        \textbf{物品} & \textbf{饥饿值} & \textbf{营养值} & \textbf{可食用} & \textbf{效果}\\
        \midrule
        蜂蜜 & 1 & 1.2 & 是 & 无 \\
        \bottomrule
    \end{tabular}
\end{table}

\subsection{石磨}

\subsubsection{面粉}

面粉,使用石磨合成。

\begin{figure}[H]
    \centering
    \begin{tikzpicture}
        \draw [recipe grid](0,0) grid (1,1);
        \draw [recipe grid](4,0) grid (6,1);
        \node (1) at (0.5,0.5) {\includegraphics[width=0.8cm,height=0.8cm]{./images/origin/wheat.png}};
        \node (2)  at (4.5,0.5) {\includegraphics[width=0.8cm,height=0.8cm]{./images/mod/flour.png}};
        \node (3)  at (5.5,0.5) {\includegraphics[width=0.8cm,height=0.8cm]{./images/mod/straw.png}};
        \begin{scope}
            \draw [-{Stealth},thick] (1.5,0.5) -- (3.5,0.5);
        \end{scope}
        \begin{scope}[xshift = 0.35cm, yshift = -0.35cm]
            \node (1sub) at (0.5,0.5) [font=\scriptsize] {3};
        \end{scope}
    \end{tikzpicture}
    \caption{面粉 UI}
\end{figure}

\begin{table}[H]
    \centering
    \caption{面粉配方}
    \setlength{\tabcolsep}{4mm}
    \begin{tabular}{c|cc|cc}
        \toprule
        \textbf{物品} & \textbf{饥饿值} & \textbf{营养值} & \textbf{可食用} & \textbf{效果}\\
        \midrule
        面粉 & 5 & 1.5 & 否 & 无 \\
        \bottomrule
    \end{tabular}
\end{table}

\subsubsection{大米}

大米,使用石磨合成。

\begin{figure}[H]
    \centering
    \begin{tikzpicture}
        \draw [recipe grid](0,0) grid (1,1);
        \draw [recipe grid](4,0) grid (6,1);
        \node (1) at (0.5,0.5) {\includegraphics[width=0.8cm,height=0.8cm]{./images/mod/rice.png}};
        \node (2)  at (4.5,0.5) {\includegraphics[width=0.8cm,height=0.8cm]{./images/mod/rice_pieces.png}};
        \node (3)  at (5.5,0.5) {\includegraphics[width=0.8cm,height=0.8cm]{./images/mod/straw.png}};
        \begin{scope}
            \draw [-{Stealth},thick] (1.5,0.5) -- (3.5,0.5);
        \end{scope}
        \begin{scope}[xshift = 0.35cm, yshift = -0.35cm]
            \node (1sub) at (0.5,0.5) [font=\scriptsize] {3};
        \end{scope}
    \end{tikzpicture}
    \caption{大米配方}
\end{figure}

\begin{table}[H]
    \centering
    \caption{大米配方}
    \setlength{\tabcolsep}{4mm}
    \begin{tabular}{c|cc|cc}
        \toprule
        \textbf{物品} & \textbf{饥饿值} & \textbf{营养值} & \textbf{可食用} & \textbf{效果}\\
        \midrule
        大米 & 5 & 1.5 & 否 & 无 \\
        \bottomrule
    \end{tabular}
\end{table}

\subsubsection{淀粉}

淀粉,使用石磨合成。

\begin{figure}[H]
    \centering
    \begin{tikzpicture}
        \draw [recipe grid](0,0) grid (1,1);
        \draw [recipe grid](4,0) grid (6,1);
        \node (1) at (0.5,0.5) {\includegraphics[width=0.8cm,height=0.8cm]{./images/mod/rice_pieces.png}};
        \node (2)  at (4.5,0.5) {\includegraphics[width=0.8cm,height=0.8cm]{./images/mod/rice_flour.png}};
        \begin{scope}
            \draw [-{Stealth},thick] (1.5,0.5) -- (3.5,0.5);
        \end{scope}
    \end{tikzpicture}
    \caption{淀粉配方}
\end{figure}

\begin{table}[H]
    \centering
    \caption{淀粉配方}
    \setlength{\tabcolsep}{4mm}
    \begin{tabular}{c|cc|cc}
        \toprule
        \textbf{物品} & \textbf{饥饿值} & \textbf{营养值} & \textbf{可食用} & \textbf{效果}\\
        \midrule
        淀粉 & 5 & 1.5 & 否 & 无 \\
        \bottomrule
    \end{tabular}
\end{table}

\subsubsection{玉米粒}

玉米粒,使用石磨合成。

\begin{figure}[H]
    \centering
    \begin{tikzpicture}
        \draw [recipe grid](0,0) grid (1,1);
        \draw [recipe grid](4,0) grid (6,1);
        \node (1) at (0.5,0.5) {\includegraphics[width=0.8cm,height=0.8cm]{./images/mod/corn.png}};
        \node (2)  at (4.5,0.5) {\includegraphics[width=0.8cm,height=0.8cm]{./images/mod/corn_pieces.png}};
        \node (3)  at (5.5,0.5) {\includegraphics[width=0.8cm,height=0.8cm]{./images/mod/straw.png}};
        \begin{scope}
            \draw [-{Stealth},thick] (1.5,0.5) -- (3.5,0.5);
        \end{scope}
        \begin{scope}[xshift = 0.35cm, yshift = -0.35cm]
            \node (1sub) at (0.5,0.5) [font=\scriptsize] {3};
        \end{scope}
    \end{tikzpicture}
    \caption{玉米粒配方}
\end{figure}

\begin{table}[H]
    \centering
    \caption{玉米粒配方}
    \setlength{\tabcolsep}{4mm}
    \begin{tabular}{c|cc|cc}
        \toprule
        \textbf{物品} & \textbf{饥饿值} & \textbf{营养值} & \textbf{可食用} & \textbf{效果}\\
        \midrule
        玉米粒 & 7 & 1.5 & 否 & 无 \\
        \bottomrule
    \end{tabular}
\end{table}

\subsubsection{玉米粉}

玉米粉,使用石磨合成。

\begin{figure}[H]
    \centering
    \begin{tikzpicture}
        \draw [recipe grid](0,0) grid (1,1);
        \draw [recipe grid](4,0) grid (6,1);
        \node (1) at (0.5,0.5) {\includegraphics[width=0.8cm,height=0.8cm]{./images/mod/corn_pieces.png}};
        \node (2)  at (4.5,0.5) {\includegraphics[width=0.8cm,height=0.8cm]{./images/mod/corn_flour.png}};
        \begin{scope}
            \draw [-{Stealth},thick] (1.5,0.5) -- (3.5,0.5);
        \end{scope}
    \end{tikzpicture}
    \caption{玉米粉配方}
\end{figure}

\begin{table}[H]
    \centering
    \caption{玉米粉配方}
    \setlength{\tabcolsep}{4mm}
    \begin{tabular}{c|cc|cc}
        \toprule
        \textbf{物品} & \textbf{饥饿值} & \textbf{营养值} & \textbf{可食用} & \textbf{效果}\\
        \midrule
        玉米粉 & 7 & 1.5 & 否 & 无 \\
        \bottomrule
    \end{tabular}
\end{table}

\subsubsection{香料}

香料,使用石磨合成。

\begin{figure}[H]
    \centering
    \begin{tikzpicture}
        \draw [recipe grid](0,0) grid (1,1);
        \draw [recipe grid](4,0) grid (6,1);
        \node (1) at (0.5,0.5) {\includegraphics[width=0.8cm,height=0.8cm]{./images/mod/herb.png}};
        \node (2)  at (4.5,0.5) {\includegraphics[width=0.8cm,height=0.8cm]{./images/mod/spices.png}};
        \node (3)  at (5.5,0.5) {\includegraphics[width=0.8cm,height=0.8cm]{./images/mod/straw.png}};
        \begin{scope}
            \draw [-{Stealth},thick] (1.5,0.5) -- (3.5,0.5);
        \end{scope}
        \begin{scope}[xshift = 0.35cm, yshift = -0.35cm]
            \node (1sub) at (0.5,0.5) [font=\scriptsize] {3};
        \end{scope}
    \end{tikzpicture}
    \caption{香料配方}
\end{figure}

\begin{table}[H]
    \centering
    \caption{香料配方}
    \setlength{\tabcolsep}{4mm}
    \begin{tabular}{c|cc|cc}
        \toprule
        \textbf{物品} & \textbf{饥饿值} & \textbf{营养值} & \textbf{可食用} & \textbf{效果}\\
        \midrule
        香料 & - & 1.5 & 否 & 无 \\
        \bottomrule
    \end{tabular}
\end{table}

\subsubsection{辣椒粉}

辣椒粉(chill\_powder),使用石磨合成。

\begin{figure}[H]
    \centering
    \begin{tikzpicture}
        \draw [recipe grid](0,0) grid (1,1);
        \draw [recipe grid](4,0) grid (6,1);
        \node (1) at (0.5,0.5) {\includegraphics[width=0.8cm,height=0.8cm]{./images/mod/pepper.png}};
        \node (2)  at (4.5,0.5) {\includegraphics[width=0.8cm,height=0.8cm]{./images/mod/chill_powder.png}};
        \node (3)  at (5.5,0.5) {\includegraphics[width=0.8cm,height=0.8cm]{./images/mod/straw.png}};
        \begin{scope}
            \draw [-{Stealth},thick] (1.5,0.5) -- (3.5,0.5);
        \end{scope}
    \end{tikzpicture}
    \caption{辣椒粉配方}
\end{figure}

\begin{table}[H]
    \centering
    \caption{辣椒粉配方}
    \setlength{\tabcolsep}{4mm}
    \begin{tabular}{c|cc|cc}
        \toprule
        \textbf{物品} & \textbf{饥饿值} & \textbf{营养值} & \textbf{可食用} & \textbf{效果}\\
        \midrule
        辣椒粉 & - & 1.5 & 否 & 无 \\
        \bottomrule
    \end{tabular}
\end{table}

\subsection{压制机}

\subsubsection{黄油}

\import{../tikz/recipes/squeezer}{butter.tex}

\begin{table}[H]
    \centering
    \caption{黄油配方}
    \setlength{\tabcolsep}{4mm}
    \begin{tabular}{c|cc|cc}
        \toprule
        \textbf{物品} & \textbf{饥饿值} & \textbf{营养值} & \textbf{可食用} & \textbf{效果}\\
        \midrule
        黄油 & 3 & 1.2 & 否 & 无 \\
        \bottomrule
    \end{tabular}
\end{table}

\subsubsection{奶酪}

\import{../tikz/recipes/squeezer}{cheese.tex}

\begin{table}[H]
    \centering
    \caption{奶酪配方}
    \setlength{\tabcolsep}{4mm}
    \begin{tabular}{c|cc|cc}
        \toprule
        \textbf{物品} & \textbf{饥饿值} & \textbf{营养值} & \textbf{可食用} & \textbf{效果}\\
        \midrule
        奶酪 & 2 & 1.2 & 否 & 无 \\
        \bottomrule
    \end{tabular}
\end{table}

\subsubsection{食用油}

\import{../tikz/recipes/squeezer}{cooking_oil.tex}

\begin{table}[H]
    \centering
    \caption{食用油配方}
    \setlength{\tabcolsep}{4mm}
    \begin{tabular}{c|cc|cc}
        \toprule
        \textbf{物品} & \textbf{饥饿值} & \textbf{营养值} & \textbf{可食用} & \textbf{效果}\\
        \midrule
        食用油 & 3 & 1.2 & 否 & 无 \\
        \bottomrule
    \end{tabular}
\end{table}

\newpage