\section{植物系统}

原版植物生长仅需要土地,光照等简单条件,且仅含一次性收获(小麦)与藤曼生长(南瓜)两种方式,较为简单。模组的植物通过脚本控制,添加了植物种植条件:生态,攀爬。增添了可多次收获植物,树木...以丰富玩法。

\subsection{粮食}

粮食的特点: 多产,提供大量饱食度,无 buff。

\subsubsection{水稻}

水稻的获取难度被设计为\textbf{中等},其特征如下:
\begin{table}[H]
    \centering
    \caption{水稻特征}
    \label{table:水稻特征}
    \setlength{\tabcolsep}{4mm}
    \begin{tabular}{c|cc|cc}
        \toprule
        \textbf{模型特征}                  & 模型高度 & 1      & 模型形状 & 井字形 \\
        \midrule
        \textbf{单方块特征}                & 生长方块 & 草方块 & 替换方块 & 空气   \\
        \midrule
        \textbf{散植特征}                  & 生成几率 & 0.1    & 生成次数 & 3      \\
        \midrule
        \multirow{2}{*}{\textbf{特征规则}} & 生成地   & 地表   & 生态     & 水系   \\
                                           & 生成条件 & 河边                       \\
        \bottomrule
    \end{tabular}
\end{table}


水稻的种植难度被设计为\textbf{简单},其种植信息如下:

\begin{table}[H]
    \centering
    \caption{水稻种植信息}
    \label{table:水稻种植信息}
    \setlength{\tabcolsep}{4mm}
    \begin{tabular}{c|cc|cc}
        \toprule
                                           & \textbf{属性} & \textbf{说明}    & \textbf{属性} & \textbf{说明}   \\
        \midrule
        \multirow{2}{*}{\textbf{种植条件}} & 土地          & 农田             & 温度          & 温带,热带      \\
                                           & 生态          & 平原,森林,水域 & 特殊          & 附近 5 格内有水 \\
        \midrule
        \multirow{2}{*}{\textbf{生长条件}} & 光照          & [11,15]          & 海拔          & [64,128]        \\
                                           & 天气          & 无               & 特殊          & 下雨发芽               \\
        \midrule
        \textbf{生长速度}                  & 速度          & 快(7tick)        & 阶段tick      & [2,2,3]         \\
        \midrule
        \multirow{2}{*}{\textbf{收获}}     & 收获次数      & 1                & 掉落物        & 水稻:3-7        \\
                                           & 饥饿度        & 2                & 营养值        & 0.4             \\
                                           & 种子 & 果实分离 \\
        \bottomrule
    \end{tabular}
\end{table}

\subsubsection{玉米}

玉米的获取难度被设计为\textbf{中等},其特征如下:
\begin{table}[H]
    \centering
    \caption{玉米特征}
    \label{table:玉米特征}
    \setlength{\tabcolsep}{4mm}
    \begin{tabular}{c|cc|cc}
        \toprule
        \textbf{模型特征}   & 模型高度 & 2      & 模型形状 & 十字 \\
        \midrule
        \textbf{单方块特征} & 生长方块 & 草方块 & 替换方块 & 空气 \\
        \midrule
        \textbf{散植特征}   & 生成几率 & 0.05   & 生成次数 & 1    \\
        \midrule
        \textbf{特征规则}   & 生成地   & 地表   & 生态     & 平原 \\
        \bottomrule
    \end{tabular}
\end{table}


玉米的种植难度被设计为\textbf{简单},其种植信息如下:

\begin{table}[H]
    \centering
    \caption{玉米种植信息}
    \label{table:玉米种植信息}
    \setlength{\tabcolsep}{4mm}
    \begin{tabular}{c|cc|cc}
        \toprule
                                           & \textbf{属性} & \textbf{说明}    & \textbf{属性} & \textbf{说明}      \\
        \midrule
        \multirow{2}{*}{\textbf{种植条件}} & 土地          & 农田,草地       & 温度          & 温带,热带,亚寒带 \\
                                           & 生态          & 平原,森林,水域 & 特殊          & -                  \\
        \midrule
        \multirow{2}{*}{\textbf{生长条件}} & 光照          & [9,15]           & 海拔          & [64,192]           \\
                                           & 天气          & 无               & 特殊          & -                  \\
        \midrule
        \textbf{生长速度}                  & 速度          & 较慢(14tick)     & 阶段tick      & [3,3,3,5]          \\
        \midrule
        \multirow{2}{*}{\textbf{收获}}     & 收获次数      & 3                & 掉落物        & 玉米:2-3           \\
                                           & 饥饿度        & 3                & 营养值        & 0.6                \\
                                           & 种子 & 果实分离 \\
        \bottomrule
    \end{tabular}
\end{table}

\subsection{蔬菜}

蔬菜的特点: 中等产量,作为辅助原材料,一般无 buff。

\subsubsection{番茄}

番茄的获取难度被设计为\textbf{中等},其特征如下:
\begin{table}[H]
    \centering
    \caption{番茄特征}
    \label{table:番茄特征}
    \setlength{\tabcolsep}{4mm}
    \begin{tabular}{c|cc|cc}
        \toprule
        \textbf{模型特征}   & 模型高度 & 1      & 模型形状 & 井字 \\
        \midrule
        \textbf{单方块特征} & 生长方块 & 草方块 & 替换方块 & 空气 \\
        \midrule
        \textbf{散植特征}   & 生成几率 & 0.05   & 生成次数 & 5    \\
        \midrule
        \textbf{特征规则}   & 生成地   & 地表   & 生态     & 平原 \\
        \bottomrule
    \end{tabular}
\end{table}


番茄的种植难度被设计为\textbf{简单},其种植信息如下:

\begin{table}[H]
    \centering
    \caption{番茄种植信息}
    \label{table:番茄种植信息}
    \setlength{\tabcolsep}{4mm}
    \begin{tabular}{c|cc|cc}
        \toprule
                                           & \textbf{属性} & \textbf{说明}    & \textbf{属性} & \textbf{说明} \\
        \midrule
        \multirow{2}{*}{\textbf{种植条件}} & 土地          & 农田,草地       & 温度          & 温带,热带    \\
                                           & 生态          & 平原,森林,水域 & 特殊          & 攀藤          \\
        \midrule
        \multirow{2}{*}{\textbf{生长条件}} & 光照          & [6,15]           & 海拔          & [64,128]      \\
                                           & 天气          & 无               & 特殊          & -             \\
        \midrule
        \textbf{生长速度}                  & 速度          & 中等(11tick)     & 阶段tick      & [3,4,4]       \\
        \midrule
        \multirow{3}{*}{\textbf{收获}}     & 收获次数      & 2                & 掉落物        & 番茄:2-5      \\
                                           & 饥饿度        & 2                & 营养值        & 0.6           \\
                                        & 种子 & 果实即为种子 \\
        \bottomrule
    \end{tabular}
\end{table}

\subsubsection{洋葱}

洋葱的获取难度被设计为\textbf{中等},其特征如下:
\begin{table}[H]
    \centering
    \caption{洋葱特征}
    \label{table:洋葱特征}
    \setlength{\tabcolsep}{4mm}
    \begin{tabular}{c|cc|cc}
        \toprule
        \textbf{模型特征}   & 模型高度 & 1      & 模型形状 & 井字 \\
        \midrule
        \textbf{单方块特征} & 生长方块 & 草方块 & 替换方块 & 空气 \\
        \midrule
        \textbf{散植特征}   & 生成几率 & 0.1   & 生成次数 & 3    \\
        \midrule
        \textbf{特征规则}   & 生成地   & 地表   & 生态     & 寒带森林 \\
        \bottomrule
    \end{tabular}
\end{table}


洋葱的种植难度被设计为\textbf{简单},其种植信息如下:

\begin{table}[H]
    \centering
    \caption{洋葱种植信息}
    \label{table:洋葱种植信息}
    \setlength{\tabcolsep}{4mm}
    \begin{tabular}{c|cc|cc}
        \toprule
                                           & \textbf{属性} & \textbf{说明}    & \textbf{属性} & \textbf{说明} \\
        \midrule
        \multirow{2}{*}{\textbf{种植条件}} & 土地          & 农田,草地       & 温度          & 温带,亚寒带    \\
                                           & 生态          & 平原,森林,水域 & 特殊          & -          \\
        \midrule
        \multirow{2}{*}{\textbf{生长条件}} & 光照          & [9,15]           & 海拔          & [64,192]      \\
                                           & 天气          & 无               & 特殊          & -             \\
        \midrule
        \textbf{生长速度}                  & 速度          & 快(7tick)     & 阶段tick      & [2,2,3]       \\
        \midrule
        \multirow{3}{*}{\textbf{收获}}     & 收获次数      & 1                & 掉落物        & 洋葱:2-5      \\
                                           & 饥饿度        & 2                & 营养值        & 0.4           \\
                                        & 种子 & 果实即为种子 \\
        \bottomrule
    \end{tabular}
\end{table}

\subsection{水果}

水果的特点: 可直接食用,可做成饮料,可提供 buff。

\subsection{配料}

配料的特点: 不提供饱食度。

\subsubsection{香草}

香草的获取难度被设计为\textbf{简单},其特征如下:
\begin{table}[H]
    \centering
    \caption{香草特征}
    \label{table:香草特征}
    \setlength{\tabcolsep}{4mm}
    \begin{tabular}{c|cc|cc}
        \toprule
        \textbf{模型特征}   & 模型高度 & 1      & 模型形状 & 井字形     \\
        \midrule
        \textbf{单方块特征} & 生长方块 & 草方块 & 替换方块 & 空气       \\
        \midrule
        \textbf{散植特征}   & 生成几率 & 0.1    & 生成次数 & 7          \\
        \midrule
        \textbf{特征规则}   & 生成地   & 地表   & 生态     & 森林,平原 \\
        \bottomrule
    \end{tabular}
\end{table}


香草的种植难度被设计为\textbf{简单},其种植信息如下:

\begin{table}[H]
    \centering
    \caption{香草种植信息}
    \label{table:香草种植信息}
    \setlength{\tabcolsep}{4mm}
    \begin{tabular}{c|cc|cc}
        \toprule
                                           & \textbf{属性} & \textbf{说明}    & \textbf{属性} & \textbf{说明}    \\
        \midrule
        \multirow{2}{*}{\textbf{种植条件}} & 土地          & 农田,草方块     & 温度          & 温带             \\
                                           & 生态          & 平原,森林,水域 & 特殊          & -                \\
        \midrule
        \multirow{2}{*}{\textbf{生长条件}} & 光照          & [9,15]           & 海拔          & [64,192]         \\
                                           & 天气          & 无               & 特殊          & -                \\
        \midrule
        \textbf{生长速度}                  & 速度          & 中等(11tick)     & 阶段tick      & [3,3,5]          \\
        \midrule
        \multirow{2}{*}{\textbf{收获}}     & 收获次数      & 1                & 掉落物        & 种子1-2,香草1-4 \\
                                           & 饥饿度        & 0                & 营养值        & 0                \\
                                           & 种子 & 果实种子分离 \\
        \bottomrule
    \end{tabular}
\end{table}

\subsubsection{辣椒}

辣椒的获取难度被设计为\textbf{中等},其特征如下:
\begin{table}[H]
    \centering
    \caption{辣椒特征}
    \label{table:辣椒特征}
    \setlength{\tabcolsep}{4mm}
    \begin{tabular}{c|cc|cc}
        \toprule
        \textbf{模型特征}                  & 模型高度 & 1      & 模型形状 & 井字形 \\
        \midrule
        \textbf{单方块特征}                & 生长方块 & 草方块 & 替换方块 & 空气   \\
        \midrule
        \textbf{散植特征}                  & 生成几率 & 0.2    & 生成次数 & 3      \\
        \midrule
        \multirow{2}{*}{\textbf{特征规则}} & 生成地   & 地表   & 生态     & 森林   \\
                                           & 生成条件 & 树下                       \\
        \bottomrule
    \end{tabular}
\end{table}


辣椒的种植难度被设计为\textbf{简单},其种植信息如下:

\begin{table}[H]
    \centering
    \caption{辣椒种植信息}
    \label{table:辣椒种植信息}
    \setlength{\tabcolsep}{4mm}
    \begin{tabular}{c|cc|cc}
        \toprule
                                           & \textbf{属性} & \textbf{说明} & \textbf{属性} & \textbf{说明} \\
        \midrule
        \multirow{2}{*}{\textbf{种植条件}} & 土地          & 农田,草方块  & 温度          & 温带          \\
                                           & 生态          & 平原,森林    & 特殊          & -             \\
        \midrule
        \multirow{2}{*}{\textbf{生长条件}} & 光照          & [9,15]        & 海拔          & [64,128]      \\
                                           & 天气          & 无            & 特殊          & -             \\
        \midrule
        \textbf{生长速度}                  & 速度          & 快(7tick)     & 阶段tick      & [2,2,3]       \\
        \midrule
        \multirow{2}{*}{\textbf{收获}}     & 收获次数      & 1             & 掉落物        & 辣椒 2-5      \\
                                           & 饥饿度        & 0             & 营养值        & 0             \\
                                           & 种子 & 果实即为种子 \\
        \bottomrule
    \end{tabular}
\end{table}

\newpage