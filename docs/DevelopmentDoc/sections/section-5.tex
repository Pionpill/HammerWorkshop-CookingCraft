\section{植物系统}

原版植物生长仅需要土地,光照等简单条件,且仅含一次性收获(小麦)与藤曼生长(南瓜)两种方式,较为简单。模组的植物通过脚本控制,添加了植物种植条件:生态,攀爬。增添了可多次收获植物,树木...以丰富玩法。

我们将植物按作用分为三类: 粮食,蔬菜,水果,调料。植物的类被决定了植物果实在合成食物的过程中承担的角色(主要原材料,调料等)。同时不同的类别有其特点,比如粮食能大量生产,提供大量饱食度。

其次,按照现实世界中的植物类型,我们将植物分为草本,禾本,藤蔓,木本,菌类,特殊。其在 MC 中的特性如下\footnote{此为大部分植物遵守的准则,也有例外,如香蕉这一大型草本植物,可收获3次}:

\begin{table}[H]
    \centering
    \caption{植物类别特性}
    \label{table:植物类别特性}
    \setlength{\tabcolsep}{4mm}
    \begin{tabular}{c|ccccc}
        \toprule
        \textbf{植物类型} & \textbf{草本} & \textbf{禾本} & \textbf{藤蔓} & \textbf{木本} & \textbf{菌类} \\
        \midrule
        生长土地          & 草地,田地    & 田地          & 田地          & 草地          & 灰化土/菌丝        \\
        生长时间          & 快            & 中等          & 中等          & 慢            & 中等          \\
        收获次数          & 1             & 1-3           & 1-5           & 特殊          & 特殊          \\
        生长方式          & 直接种植      & 直接种植      & 攀藤          & 种植生长      & 直接种植      \\
        \bottomrule
    \end{tabular}
\end{table}

此外,针对植物的获取与种植条件,设置难度如下。

\begin{table}[H]
    \centering
    \caption{植物获取}
    \label{table:植物获取}
    \setlength{\tabcolsep}{4mm}
    \begin{tabular}{c|ccc}
        \toprule
        \textbf{获取难度} & \textbf{表示}                                  & \textbf{说明}                   & \textbf{例子} \\
        \midrule
        简单              & $\bigstar$                                     & 世界大概率遇见                  & 香草          \\
        中等              & $\bigstar\bigstar$                             & 世界中等概率遇见                & 辣椒          \\
        困难              & $\bigstar \bigstar \bigstar$                   & 世界小概率遇见,怪物掉落        & 水稻          \\
        极难              & $\bigstar \bigstar \bigstar \bigstar$          & 条件极为严苛或通过购买/宝箱获得 & 雪莲          \\
        传奇              & $\bigstar \bigstar \bigstar \bigstar \bigstar$ & 极其特殊的获取方式              & 暂无          \\
        \midrule
        \textbf{种植难度} & \textbf{表示}                                  & \textbf{说明}                   & \textbf{例子} \\
        \midrule
        简单              & $\bigstar$                                     & 生态等条件宽裕                  & 香草          \\
        中等              & $\bigstar\bigstar$                             & 需要特殊的条件,但容易达到      & 水稻          \\
        困难              & $\bigstar \bigstar \bigstar$                   & 种植生态较少                    & 香蕉          \\
        极难              & $\bigstar \bigstar \bigstar \bigstar$          & 严苛的种植条件                  & 雪莲          \\
        传奇              & $\bigstar \bigstar \bigstar \bigstar \bigstar$ & 无法种植或极难达到              & 暂无          \\
        \bottomrule
    \end{tabular}
\end{table}

在我们的模组中,设计的所有植物如下:

\begin{table}[H]
    \centering
    \caption{模组植物}
    \label{table:模组植物}
    \setlength{\tabcolsep}{4mm}
    \begin{tabular}{l|ll|ll|cl}
        \toprule
        \textbf{类别} &\textbf{植物名} &  \textbf{种别} & \textbf{获取难度}  & \textbf{种植难度}  & \textbf{实装} & \textbf{备注} \\
        \midrule
        \multirow{7}{*}{\textbf{粮食}}& 水稻   & 草本  & +++    & ++ & √ &         \\
        & 青稞 & 草本 & & & & \\
        & 玉米   & 禾本  & ++    & + & √ &         \\
        & 黄豆   & 草本  & ++    & ++ & - &         \\
        & 红豆   & 草本  & ++    & ++ & - &         \\
        & 绿豆   & 草本  & ++    & ++ & - &         \\
        & 红薯   & 草本  & +    & + &  &         \\
        \midrule
        \multirow{9}{*}{\textbf{蔬菜}}& 番茄   & 藤蔓  & +++    & + &  √ &  需要修正  \\
        & 洋葱   & 草本  & ++    & + & √ &         \\
        & 菜椒   & 草本  & ++    & + & -  &        \\
        & 豌豆   & 草本  & ++    & + &   &        \\
        & 茄子   & 草本  & ++    & + & - &         \\
        & 白菜   & 草本  & +++    & ++ &   &        \\
        & 白萝卜   & 草本  &    &  & - &        \\
        & 卷心菜   & 草本  & +++    & +++ &   &        \\
        & 雪莲   & 草本  & +++++    & +++++ &  &         \\
        \midrule
        \multirow{6}{*}{\textbf{菌类}}& 口菇   & 菌类  & +++    & ++ & - &   \\
        & 香菇   & 菌类  & +++    & +++ & - &        \\
        & 金针菇   & 菌类  & ++    & ++ & - &        \\
        & 松茸   & 菌类  & ++++    & ++++ & -  &       \\
        & 松露   & 菌类  & ++++    & +++++ &  & 无法养殖    \\
        & 黑松露   & 菌类  & +++++    & +++++ &  & 无法养殖   \\
        \midrule
        \multirow{6}{*}{\textbf{水果}}& 香蕉 & 大型草本 & +++ & +++ & √ & \\
        & 菠萝 & 大型草本 &  &  &  & \\
        & 火龙果 & 多肉 &  &  &  & \\
        & 哈密瓜 & 草本 &  &  &  & \\
        & 雪莲果 & 草本 &  &  &  & \\
        & 百香果 & 藤蔓 &  &  &  & \\
        \midrule
        \multirow{4}{*}{\textbf{调料}}& 香草 & 草本 & + & + & √ &\\
        & 辣椒 & 草本  & ++ & + & √ &\\
        & 油菜花 & 草本 & + & + &  &\\
        & 薄荷 & 草本 &  &  &  &\\
        \bottomrule
    \end{tabular}
\end{table}

\subsection{粮食}

粮食的特点: 多产,提供大量饱食度,无 buff。

\subsubsection{水稻}

水稻的获取难度被设计为\textbf{困难},其特征如下:
\begin{table}[H]
    \centering
    \caption{水稻特征}
    \label{table:水稻特征}
    \setlength{\tabcolsep}{4mm}
    \begin{tabular}{c|cc|cc}
        \toprule
        \textbf{模型特征}                  & 模型高度 & 1      & 模型形状 & 井字形 \\
        \midrule
        \textbf{单方块特征}                & 生长方块 & 草方块 & 替换方块 & 空气   \\
        \midrule
        \textbf{散植特征}                  & 生成几率 & 0.8    & 生成次数 & 3      \\
        \midrule
        \multirow{2}{*}{\textbf{特征规则}} & 生成地   & 地表   & 生态     & 水系   \\
                                           & 生成条件 & 河边                       \\
        \bottomrule
    \end{tabular}
\end{table}


水稻的种植难度被设计为\textbf{中等},其种植信息如下:

\begin{table}[H]
    \centering
    \caption{水稻种植信息}
    \label{table:水稻种植信息}
    \setlength{\tabcolsep}{4mm}
    \begin{tabular}{c|cc|cc}
        \toprule
                                           & \textbf{属性} & \textbf{说明}   & \textbf{属性} & \textbf{说明} \\
        \midrule
        \multirow{4}{*}{\textbf{种植条件}} & 土地          & 农田            & 生态(温度)    & 温带,热带    \\
                                           & 生态(降雨)    & 中,高          & 生态(海拔)    & 中            \\
                                           & 生态(土地)    & 草地            & 生态(特殊)    & 无            \\
                                           & 特殊          & 附近 3 格内有水                                 \\
        \midrule
        \multirow{2}{*}{\textbf{生长条件}} & 光照          & [11,15]         & 海拔          & [64,128]      \\
                                           & 天气          & 无              & 特殊          & 下雨发芽      \\
        \midrule
        \textbf{生长速度}                  & 速度          & 快(7tick)       & 阶段tick      & [2,2,3]       \\
        \midrule
        \multirow{3}{*}{\textbf{收获}}     & 收获次数      & 1               & 掉落物        & 水稻:3-7      \\
                                           & 饥饿度        & 2               & 营养值        & 0.4           \\
                                           & 种子          & 果实分离        & 生吃          & 不可          \\
        \bottomrule
    \end{tabular}
\end{table}

\subsubsection{玉米}

玉米的获取难度被设计为\textbf{中等},其特征如下:
\begin{table}[H]
    \centering
    \caption{玉米特征}
    \label{table:玉米特征}
    \setlength{\tabcolsep}{4mm}
    \begin{tabular}{c|cc|cc}
        \toprule
        \textbf{模型特征}   & 模型高度 & 2      & 模型形状 & X字  \\
        \midrule
        \textbf{单方块特征} & 生长方块 & 草方块 & 替换方块 & 空气 \\
        \midrule
        \textbf{散植特征}   & 生成几率 & 0.05   & 生成次数 & 1    \\
        \midrule
        \textbf{特征规则}   & 生成地   & 地表   & 生态     & 平原 \\
        \bottomrule
    \end{tabular}
\end{table}


玉米的种植难度被设计为\textbf{简单},其种植信息如下:

\begin{table}[H]
    \centering
    \caption{玉米种植信息}
    \label{table:玉米种植信息}
    \setlength{\tabcolsep}{4mm}
    \begin{tabular}{c|cc|cc}
        \toprule
                                           & \textbf{属性} & \textbf{说明} & \textbf{属性} & \textbf{说明} \\
        \midrule
        \multirow{4}{*}{\textbf{种植条件}} & 土地          & 农田          & 生态(温度)    & 温带,热带    \\
                                           & 生态(降雨)    & 中            & 生态(海拔)    & 中            \\
                                           & 生态(土地)    & 草地          & 生态(特殊)    & 针叶林        \\
                                           & 特殊          & -                                             \\
        \midrule
        \multirow{2}{*}{\textbf{生长条件}} & 光照          & [9,15]        & 海拔          & [64,192]      \\
                                           & 天气          & 无            & 特殊          & -             \\
        \midrule
        \textbf{生长速度}                  & 速度          & 较慢(14tick)  & 阶段tick      & [3,3,3,5]     \\
        \midrule
        \multirow{2}{*}{\textbf{收获}}     & 收获次数      & 3             & 掉落物        & 玉米:2-3      \\
                                           & 饥饿度        & 3             & 营养值        & 0.6           \\
                                           & 种子          & 果实分离      & 生吃          & 不可          \\
        \bottomrule
    \end{tabular}
\end{table}

\subsubsection{黄豆}

黄豆的获取难度被设计为\textbf{中等},其特征如下:
\begin{table}[H]
    \centering
    \caption{黄豆特征}
    \label{table:黄豆特征}
    \setlength{\tabcolsep}{4mm}
    \begin{tabular}{c|cc|cc}
        \toprule
        \textbf{模型特征}                  & 模型高度 & 1      & 模型形状 & 井字形 \\
        \midrule
        \textbf{单方块特征}                & 生长方块 & 草方块 & 替换方块 & 空气   \\
        \midrule
        \textbf{散植特征}                  & 生成几率 & 0.1    & 生成次数 & 3      \\
        \midrule
        \multirow{2}{*}{\textbf{特征规则}} & 生成地   & 地表   & 生态     & 温带森林   \\
                                           & 生成条件 & -              \\
        \bottomrule
    \end{tabular}
\end{table}


黄豆的种植难度被设计为\textbf{中等},其种植信息如下:

\begin{table}[H]
    \centering
    \caption{黄豆种植信息}
    \label{table:黄豆种植信息}
    \setlength{\tabcolsep}{4mm}
    \begin{tabular}{c|cc|cc}
        \toprule
                                           & \textbf{属性} & \textbf{说明}   & \textbf{属性} & \textbf{说明} \\
        \midrule
        \multirow{4}{*}{\textbf{种植条件}} & 土地          & 农田            & 生态(温度)    & 温带,亚寒带    \\
                                           & 生态(降雨)    & 中,高          & 生态(海拔)    & 中            \\
                                           & 生态(土地)    & 草地            & 生态(特殊)    & 无            \\
                                           & 特殊          & -                                 \\
        \midrule
        \multirow{2}{*}{\textbf{生长条件}} & 光照          & [9,15]         & 海拔          & [64,128]      \\
                                           & 天气          & 无              & 特殊          & -      \\
        \midrule
        \textbf{生长速度}                  & 速度          & 中等(10tick)       & 阶段tick      & [3,3,4]       \\
        \midrule
        \multirow{3}{*}{\textbf{收获}}     & 收获次数      & 1               & 掉落物        & 黄豆:2-5      \\
                                           & 饥饿度        & 3               & 营养值        & 0.6           \\
                                           & 种子          & 果实分离        & 生吃          & 不可          \\
        \bottomrule
    \end{tabular}
\end{table}

\subsubsection{绿豆}

绿豆的获取难度被设计为\textbf{中等},其特征如下:
\begin{table}[H]
    \centering
    \caption{绿豆特征}
    \label{table:绿豆特征}
    \setlength{\tabcolsep}{4mm}
    \begin{tabular}{c|cc|cc}
        \toprule
        \textbf{模型特征}                  & 模型高度 & 1      & 模型形状 & 井字形 \\
        \midrule
        \textbf{单方块特征}                & 生长方块 & 草方块 & 替换方块 & 空气   \\
        \midrule
        \textbf{散植特征}                  & 生成几率 & 0.1    & 生成次数 & 3      \\
        \midrule
        \multirow{2}{*}{\textbf{特征规则}} & 生成地   & 地表   & 生态     & 温带森林   \\
                                           & 生成条件 & -              \\
        \bottomrule
    \end{tabular}
\end{table}


绿豆的种植难度被设计为\textbf{中等},其种植信息如下:

\begin{table}[H]
    \centering
    \caption{绿豆种植信息}
    \label{table:绿豆种植信息}
    \setlength{\tabcolsep}{4mm}
    \begin{tabular}{c|cc|cc}
        \toprule
                                           & \textbf{属性} & \textbf{说明}   & \textbf{属性} & \textbf{说明} \\
        \midrule
        \multirow{4}{*}{\textbf{种植条件}} & 土地          & 农田            & 生态(温度)    & 温带,亚寒带,寒带    \\
                                           & 生态(降雨)    & 中,高          & 生态(海拔)    & 中            \\
                                           & 生态(土地)    & 草地            & 生态(特殊)    & 无            \\
                                           & 特殊          & -                                 \\
        \midrule
        \multirow{2}{*}{\textbf{生长条件}} & 光照          & [9,15]         & 海拔          & [64,128]      \\
                                           & 天气          & 无              & 特殊          & -      \\
        \midrule
        \textbf{生长速度}                  & 速度          & 中等(10tick)       & 阶段tick      & [3,3,4]       \\
        \midrule
        \multirow{3}{*}{\textbf{收获}}     & 收获次数      & 1               & 掉落物        & 绿豆:2-5      \\
                                           & 饥饿度        & 3               & 营养值        & 0.6           \\
                                           & 种子          & 果实分离        & 生吃          & 不可          \\
        \bottomrule
    \end{tabular}
\end{table}

\subsubsection{红豆}

红豆的获取难度被设计为\textbf{中等},其特征如下:
\begin{table}[H]
    \centering
    \caption{红豆特征}
    \label{table:红豆特征}
    \setlength{\tabcolsep}{4mm}
    \begin{tabular}{c|cc|cc}
        \toprule
        \textbf{模型特征}                  & 模型高度 & 1      & 模型形状 & 井字形 \\
        \midrule
        \textbf{单方块特征}                & 生长方块 & 草方块 & 替换方块 & 空气   \\
        \midrule
        \textbf{散植特征}                  & 生成几率 & 0.1    & 生成次数 & 3      \\
        \midrule
        \multirow{2}{*}{\textbf{特征规则}} & 生成地   & 地表   & 生态     & 温带森林   \\
                                           & 生成条件 & -              \\
        \bottomrule
    \end{tabular}
\end{table}


红豆的种植难度被设计为\textbf{中等},其种植信息如下:

\begin{table}[H]
    \centering
    \caption{红豆种植信息}
    \label{table:红豆种植信息}
    \setlength{\tabcolsep}{4mm}
    \begin{tabular}{c|cc|cc}
        \toprule
                                           & \textbf{属性} & \textbf{说明}   & \textbf{属性} & \textbf{说明} \\
        \midrule
        \multirow{4}{*}{\textbf{种植条件}} & 土地          & 农田            & 生态(温度)    & 温带,热带    \\
                                           & 生态(降雨)    & 中,高          & 生态(海拔)    & 中            \\
                                           & 生态(土地)    & 草地            & 生态(特殊)    & 无            \\
                                           & 特殊          & -                                 \\
        \midrule
        \multirow{2}{*}{\textbf{生长条件}} & 光照          & [9,15]         & 海拔          & [64,128]      \\
                                           & 天气          & 无              & 特殊          & -      \\
        \midrule
        \textbf{生长速度}                  & 速度          & 中等(10tick)       & 阶段tick      & [3,3,4]       \\
        \midrule
        \multirow{3}{*}{\textbf{收获}}     & 收获次数      & 1               & 掉落物        & 红豆:2-5      \\
                                           & 饥饿度        & 3               & 营养值        & 0.6           \\
                                           & 种子          & 果实分离        & 生吃          & 不可          \\
        \bottomrule
    \end{tabular}
\end{table}

\subsubsection{红薯}

红薯的获取难度被设计为\textbf{简单},其特征如下:
\begin{table}[H]
    \centering
    \caption{红薯特征}
    \label{table:红薯特征}
    \setlength{\tabcolsep}{4mm}
    \begin{tabular}{c|cc|cc}
        \toprule
        \textbf{模型特征}                  & 模型高度 & 1      & 模型形状 & 井字形 \\
        \midrule
        \textbf{单方块特征}                & 生长方块 & 草方块 & 替换方块 & 空气   \\
        \midrule
        \textbf{散植特征}                  & 生成几率 & 0.1    & 生成次数 & 4      \\
        \midrule
        \multirow{2}{*}{\textbf{特征规则}} & 生成地   & 地表   & 生态     & 温带草原   \\
                                           & 生成条件 & -              \\
        \bottomrule
    \end{tabular}
\end{table}


红薯的种植难度被设计为\textbf{简单},其种植信息如下:

\begin{table}[H]
    \centering
    \caption{红薯种植信息}
    \label{table:红薯种植信息}
    \setlength{\tabcolsep}{4mm}
    \begin{tabular}{c|cc|cc}
        \toprule
                                           & \textbf{属性} & \textbf{说明}   & \textbf{属性} & \textbf{说明} \\
        \midrule
        \multirow{4}{*}{\textbf{种植条件}} & 土地          & 农田            & 生态(温度)    & 温带,寒带    \\
                                           & 生态(降雨)    & 中,高          & 生态(海拔)    & 中            \\
                                           & 生态(土地)    & 草地            & 生态(特殊)    & 无            \\
                                           & 特殊          & -                                 \\
        \midrule
        \multirow{2}{*}{\textbf{生长条件}} & 光照          & [11,15]         & 海拔          & [64,256]      \\
                                           & 天气          & 无              & 特殊          & -      \\
        \midrule
        \textbf{生长速度}                  & 速度          & 较慢(13tick)       & 阶段tick      & [4,4,5]       \\
        \midrule
        \multirow{3}{*}{\textbf{收获}}     & 收获次数      & 1               & 掉落物        & 红薯:2-5      \\
                                           & 饥饿度        & 4               & 营养值        & 0.6           \\
                                           & 种子          & 果实即种子        & 生吃          & 不可       \\
        \bottomrule
    \end{tabular}
\end{table}

\subsection{蔬菜}

蔬菜的特点: 中等产量,作为辅助原材料。

\subsubsection{番茄}

番茄的获取难度被设计为\textbf{困难},其特征如下:
\begin{table}[H]
    \centering
    \caption{番茄特征}
    \label{table:番茄特征}
    \setlength{\tabcolsep}{4mm}
    \begin{tabular}{c|cc|cc}
        \toprule
        \textbf{模型特征}   & 模型高度 & 1      & 模型形状 & 井字 \\
        \midrule
        \textbf{单方块特征} & 生长方块 & 草方块 & 替换方块 & 空气 \\
        \midrule
        \textbf{散植特征}   & 生成几率 & -      & 生成次数 & -    \\
        \midrule
        \textbf{特征规则}   & 生成地   & 地表   & 生态     & 平原 \\
        \bottomrule
    \end{tabular}
\end{table}


番茄的种植难度被设计为\textbf{简单},其种植信息如下:

\begin{table}[H]
    \centering
    \caption{番茄种植信息}
    \label{table:番茄种植信息}
    \setlength{\tabcolsep}{4mm}
    \begin{tabular}{c|cc|cc}
        \toprule
                                           & \textbf{属性} & \textbf{说明} & \textbf{属性} & \textbf{说明} \\
        \midrule
        \multirow{4}{*}{\textbf{种植条件}} & 土地          & 篱笆          & 生态(温度)    & 温带,热带    \\
                                           & 生态(降雨)    & 中            & 生态(海拔)    & 中            \\
                                           & 生态(土地)    & 草地          & 生态(特殊)    & -             \\
                                           & 特殊          & -                                             \\
        \midrule
        \multirow{2}{*}{\textbf{生长条件}} & 光照          & [6,15]        & 海拔          & [64,128]      \\
                                           & 天气          & 无            & 特殊          & -             \\
        \midrule
        \textbf{生长速度}                  & 速度          & 中等(11tick)  & 阶段tick      & [3,4,4]       \\
        \midrule
        \multirow{3}{*}{\textbf{收获}}     & 收获次数      & 2             & 掉落物        & 番茄:2-5      \\
                                           & 饥饿度        & 2             & 营养值        & 0.6           \\
                                           & 种子          & 果实即为种子  & 生吃          & 可            \\
        \bottomrule
    \end{tabular}
\end{table}

\subsubsection{洋葱}

洋葱的获取难度被设计为\textbf{中等},其特征如下:
\begin{table}[H]
    \centering
    \caption{洋葱特征}
    \label{table:洋葱特征}
    \setlength{\tabcolsep}{4mm}
    \begin{tabular}{c|cc|cc}
        \toprule
        \textbf{模型特征}   & 模型高度 & 1      & 模型形状 & 井字     \\
        \midrule
        \textbf{单方块特征} & 生长方块 & 草方块 & 替换方块 & 空气     \\
        \midrule
        \textbf{散植特征}   & 生成几率 & 0.05   & 生成次数 & 3        \\
        \midrule
        \textbf{特征规则}   & 生成地   & 地表   & 生态     & 寒带森林 \\
        \bottomrule
    \end{tabular}
\end{table}


洋葱的种植难度被设计为\textbf{简单},其种植信息如下:

\begin{table}[H]
    \centering
    \caption{洋葱种植信息}
    \label{table:洋葱种植信息}
    \setlength{\tabcolsep}{4mm}
    \begin{tabular}{c|cc|cc}
        \toprule
                                           & \textbf{属性} & \textbf{说明} & \textbf{属性} & \textbf{说明} \\
        \midrule
        \multirow{4}{*}{\textbf{种植条件}} & 土地          & 农田          & 生态(温度)    & 温带,寒带    \\
                                           & 生态(降雨)    & 中,低        & 生态(海拔)    & 中            \\
                                           & 生态(土地)    & 草地          & 生态(特殊)    & -             \\
                                           & 特殊          & -                                             \\
        \midrule
        \multirow{2}{*}{\textbf{生长条件}} & 光照          & [9,15]        & 海拔          & [64,192]      \\
                                           & 天气          & 无            & 特殊          & -             \\
        \midrule
        \textbf{生长速度}                  & 速度          & 快(7tick)     & 阶段tick      & [2,2,3]       \\
        \midrule
        \multirow{3}{*}{\textbf{收获}}     & 收获次数      & 1             & 掉落物        & 洋葱:2-5      \\
                                           & 饥饿度        & 2             & 营养值        & 0.4           \\
                                           & 种子          & 果实即为种子  & 生吃          & 可(失明30s)   \\
        \bottomrule
    \end{tabular}
\end{table}

\subsubsection{菜椒}

菜椒的获取难度被设计为\textbf{中等},其特征如下:
\begin{table}[H]
    \centering
    \caption{菜椒特征}
    \label{table:菜椒特征}
    \setlength{\tabcolsep}{4mm}
    \begin{tabular}{c|cc|cc}
        \toprule
        \textbf{模型特征}   & 模型高度 & 1      & 模型形状 & 井字     \\
        \midrule
        \textbf{单方块特征} & 生长方块 & 草方块 & 替换方块 & 空气     \\
        \midrule
        \textbf{散植特征}   & 生成几率 & 0.05   & 生成次数 & 2        \\
        \midrule
        \textbf{特征规则}   & 生成地   & 地表   & 生态     & 温带森林 \\
        \bottomrule
    \end{tabular}
\end{table}


菜椒的种植难度被设计为\textbf{简单},其种植信息如下:

\begin{table}[H]
    \centering
    \caption{菜椒种植信息}
    \label{table:菜椒种植信息}
    \setlength{\tabcolsep}{4mm}
    \begin{tabular}{c|cc|cc}
        \toprule
                                           & \textbf{属性} & \textbf{说明} & \textbf{属性} & \textbf{说明} \\
        \midrule
        \multirow{4}{*}{\textbf{种植条件}} & 土地          & 农田          & 生态(温度)    & 温带    \\
                                           & 生态(降雨)    & 中,高        & 生态(海拔)    & 中            \\
                                           & 生态(土地)    & 草地          & 生态(特殊)    & -             \\
                                           & 特殊          & -                                             \\
        \midrule
        \multirow{2}{*}{\textbf{生长条件}} & 光照          & [9,15]        & 海拔          & [64,128]      \\
                                           & 天气          & 无            & 特殊          & -             \\
        \midrule
        \textbf{生长速度}                  & 速度          & 快(8tick)     & 阶段tick      & [2,3,3]       \\
        \midrule
        \multirow{3}{*}{\textbf{收获}}     & 收获次数      & 1             & 掉落物        & 菜椒:3-5      \\
                                           & 饥饿度        & 2             & 营养值        & 0.4           \\
                                           & 种子          & 果实即为种子  & 生吃          & 可   \\
        \bottomrule
    \end{tabular}
\end{table}

\subsubsection{豌豆}

豌豆的获取难度被设计为\textbf{中等},其特征如下:
\begin{table}[H]
    \centering
    \caption{豌豆特征}
    \label{table:豌豆特征}
    \setlength{\tabcolsep}{4mm}
    \begin{tabular}{c|cc|cc}
        \toprule
        \textbf{模型特征}   & 模型高度 & 1      & 模型形状 & 井字     \\
        \midrule
        \textbf{单方块特征} & 生长方块 & 草方块 & 替换方块 & 空气     \\
        \midrule
        \textbf{散植特征}   & 生成几率 & 0.05   & 生成次数 & 3        \\
        \midrule
        \textbf{特征规则}   & 生成地   & 地表   & 生态     & 温带森林 \\
        \bottomrule
    \end{tabular}
\end{table}


豌豆的种植难度被设计为\textbf{简单},其种植信息如下:

\begin{table}[H]
    \centering
    \caption{豌豆种植信息}
    \label{table:豌豆种植信息}
    \setlength{\tabcolsep}{4mm}
    \begin{tabular}{c|cc|cc}
        \toprule
                                           & \textbf{属性} & \textbf{说明} & \textbf{属性} & \textbf{说明} \\
        \midrule
        \multirow{4}{*}{\textbf{种植条件}} & 土地          & 农田          & 生态(温度)    & 温带,亚寒带    \\
                                           & 生态(降雨)    & 中,低        & 生态(海拔)    & 中,中高            \\
                                           & 生态(土地)    & 草地          & 生态(特殊)    & -             \\
                                           & 特殊          & -                                             \\
        \midrule
        \multirow{2}{*}{\textbf{生长条件}} & 光照          & [9,15]        & 海拔          & [64,192]      \\
                                           & 天气          & 无            & 特殊          & -             \\
        \midrule
        \textbf{生长速度}                  & 速度          & 快(8tick)     & 阶段tick      & [2,3,3]       \\
        \midrule
        \multirow{3}{*}{\textbf{收获}}     & 收获次数      & 1             & 掉落物        & 豌豆:3-5      \\
                                           & 饥饿度        & 2             & 营养值        & 0.4           \\
                                           & 种子          & 果实即为种子  & 生吃          & 可   \\
        \bottomrule
    \end{tabular}
\end{table}

\subsubsection{茄子}

茄子的获取难度被设计为\textbf{中等},其特征如下:
\begin{table}[H]
    \centering
    \caption{茄子特征}
    \label{table:茄子特征}
    \setlength{\tabcolsep}{4mm}
    \begin{tabular}{c|cc|cc}
        \toprule
        \textbf{模型特征}   & 模型高度 & 1      & 模型形状 & 井字     \\
        \midrule
        \textbf{单方块特征} & 生长方块 & 草方块 & 替换方块 & 空气     \\
        \midrule
        \textbf{散植特征}   & 生成几率 & 0.02   & 生成次数 & 1        \\
        \midrule
        \textbf{特征规则}   & 生成地   & 地表   & 生态     & 温带,热带草原 \\
        \bottomrule
    \end{tabular}
\end{table}


茄子的种植难度被设计为\textbf{简单},其种植信息如下:

\begin{table}[H]
    \centering
    \caption{茄子种植信息}
    \label{table:茄子种植信息}
    \setlength{\tabcolsep}{4mm}
    \begin{tabular}{c|cc|cc}
        \toprule
                                           & \textbf{属性} & \textbf{说明} & \textbf{属性} & \textbf{说明} \\
        \midrule
        \multirow{4}{*}{\textbf{种植条件}} & 土地          & 农田          & 生态(温度)    & 温带,热带    \\
                                           & 生态(降雨)    & 中,低        & 生态(海拔)    & 中            \\
                                           & 生态(土地)    & 草地          & 生态(特殊)    & -             \\
                                           & 特殊          & -                                             \\
        \midrule
        \multirow{2}{*}{\textbf{生长条件}} & 光照          & [11,15]        & 海拔          & [64,128]      \\
                                           & 天气          & 无            & 特殊          & -             \\
        \midrule
        \textbf{生长速度}                  & 速度          & 中等(11tick)     & 阶段tick      & [3,3,4]       \\
        \midrule
        \multirow{3}{*}{\textbf{收获}}     & 收获次数      & 1             & 掉落物        & 茄子:2-5      \\
                                           & 饥饿度        & 3             & 营养值        & 0.4           \\
                                           & 种子          & 果实即为种子  & 生吃          & 可(30\%概率中毒 10s) \\
        \bottomrule
    \end{tabular}
\end{table}

\subsubsection{白菜}

白菜的获取难度被设计为\textbf{困难},其特征如下:
\begin{table}[H]
    \centering
    \caption{白菜特征}
    \label{table:白菜特征}
    \setlength{\tabcolsep}{4mm}
    \begin{tabular}{c|cc|cc}
        \toprule
        \textbf{模型特征}   & 模型高度 & 1      & 模型形状 & 井字     \\
        \midrule
        \textbf{单方块特征} & 生长方块 & 草方块 & 替换方块 & 空气     \\
        \midrule
        \textbf{散植特征}   & 生成几率 & 0.02   & 生成次数 & 1        \\
        \midrule
        \textbf{特征规则}   & 生成地   & 地表/雪下   & 生态     & 亚寒带,寒带 \\
        \bottomrule
    \end{tabular}
\end{table}


白菜的种植难度被设计为\textbf{中等},其种植信息如下:

\begin{table}[H]
    \centering
    \caption{白菜种植信息}
    \label{table:白菜种植信息}
    \setlength{\tabcolsep}{4mm}
    \begin{tabular}{c|cc|cc}
        \toprule
                                           & \textbf{属性} & \textbf{说明} & \textbf{属性} & \textbf{说明} \\
        \midrule
        \multirow{4}{*}{\textbf{种植条件}} & 土地          & 农田          & 生态(温度)    & 温带,寒带,亚寒带    \\
                                           & 生态(降雨)    & 中,低        & 生态(海拔)    & 中            \\
                                           & 生态(土地)    & 草地          & 生态(特殊)    & -             \\
                                           & 特殊          & 下雨发芽         \\
        \midrule
        \multirow{2}{*}{\textbf{生长条件}} & 光照          & [11,15]        & 海拔          & [64,192]      \\
                                           & 天气          & 无            & 特殊          & -             \\
        \midrule
        \textbf{生长速度}                  & 速度          & 较慢(16tick)     & 阶段tick      & [5,5,6]       \\
        \midrule
        \multirow{3}{*}{\textbf{收获}}     & 收获次数      & 1             & 掉落物        & 白菜:2-5      \\
                                           & 饥饿度        & 3             & 营养值        & 0.4           \\
                                           & 种子          & 果实即为种子  & 生吃          & 可 \\
        \bottomrule
    \end{tabular}
\end{table}

\subsubsection{卷心菜}

卷心菜的获取难度被设计为\textbf{困难},其特征如下:
\begin{table}[H]
    \centering
    \caption{卷心菜特征}
    \label{table:卷心菜特征}
    \setlength{\tabcolsep}{4mm}
    \begin{tabular}{c|cc|cc}
        \toprule
        \textbf{模型特征}   & 模型高度 & 1      & 模型形状 & 井字     \\
        \midrule
        \textbf{单方块特征} & 生长方块 & 草方块 & 替换方块 & 空气     \\
        \midrule
        \textbf{散植特征}   & 生成几率 & 0.02   & 生成次数 & 1        \\
        \midrule
        \textbf{特征规则}   & 生成地   & 地表/雪下   & 生态     & 亚寒带,寒带 \\
        \bottomrule
    \end{tabular}
\end{table}


卷心菜的种植难度被设计为\textbf{困难},其种植信息如下:

\begin{table}[H]
    \centering
    \caption{卷心菜种植信息}
    \label{table:卷心菜种植信息}
    \setlength{\tabcolsep}{4mm}
    \begin{tabular}{c|cc|cc}
        \toprule
                                           & \textbf{属性} & \textbf{说明} & \textbf{属性} & \textbf{说明} \\
        \midrule
        \multirow{4}{*}{\textbf{种植条件}} & 土地          & 农田          & 生态(温度)    & 寒带,亚寒带    \\
                                           & 生态(降雨)    & 中,低        & 生态(海拔)    & 中            \\
                                           & 生态(土地)    & 草地          & 生态(特殊)    & -             \\
                                           & 特殊          & 下雨发芽                              \\
        \midrule
        \multirow{2}{*}{\textbf{生长条件}} & 光照          & [11,15]        & 海拔          & [64,192]      \\
                                           & 天气          & 无            & 特殊          & -             \\
        \midrule
        \textbf{生长速度}                  & 速度          & 较慢(14tick)     & 阶段tick      & [4,5,5]       \\
        \midrule
        \multirow{3}{*}{\textbf{收获}}     & 收获次数      & 1             & 掉落物        & 卷心菜:2-5      \\
                                           & 饥饿度        & 3             & 营养值        & 0.4           \\
                                           & 种子          & 果实即为种子  & 生吃          & 可 \\
        \bottomrule
    \end{tabular}
\end{table}

\subsubsection{雪莲}

雪莲的获取难度被设计为\textbf{传奇},其特征如下:
\begin{table}[H]
    \centering
    \caption{雪莲特征}
    \label{table:雪莲特征}
    \setlength{\tabcolsep}{4mm}
    \begin{tabular}{c|cc|cc}
        \toprule
        \textbf{模型特征}   & 模型高度 & 1      & 模型形状 & 井字     \\
        \midrule
        \textbf{单方块特征} & 生长方块 & 草方块 & 替换方块 & 空气     \\
        \midrule
        \textbf{散植特征}   & 生成几率 & 0.01   & 生成次数 & 1        \\
        \midrule
        \textbf{特征规则}   & 生成地   & 地表/雪下   & 生态     & 寒带 \\
        \bottomrule
    \end{tabular}
\end{table}


雪莲的种植难度被设计为\textbf{传奇},其种植信息如下:

\begin{table}[H]
    \centering
    \caption{雪莲种植信息}
    \label{table:雪莲种植信息}
    \setlength{\tabcolsep}{4mm}
    \begin{tabular}{c|cc|cc}
        \toprule
                                           & \textbf{属性} & \textbf{说明} & \textbf{属性} & \textbf{说明} \\
        \midrule
        \multirow{4}{*}{\textbf{种植条件}} & 土地          & 草地          & 生态(温度)    & 寒带    \\
                                           & 生态(降雨)    & 中,低        & 生态(海拔)    & 高      \\
                                           & 生态(土地)    & 草地          & 生态(特殊)    & -         \\
                                           & 特殊          & -                \\
        \midrule
        \multirow{2}{*}{\textbf{生长条件}} & 光照          & [6,15]        & 海拔          & [128,256]      \\
                                           & 天气          & 降雨            & 特殊          & -             \\
        \midrule
        \textbf{生长速度}                  & 速度          & 极慢(39tick)     & 阶段tick      & [12,13,14]       \\
        \midrule
        \multirow{3}{*}{\textbf{收获}}     & 收获次数      & 1             & 掉落物        & 雪莲:2      \\
                                           & 饥饿度        & 2             & 营养值        & 0.6           \\
                                           & 种子          & 果实即为种子  & 生吃          & 不可 \\
        \bottomrule
    \end{tabular}
\end{table}

\subsection{菌类}

菌类也是一种辅助材料,在过亮的地方种植会死亡。菌类在生长过程中会尝试蔓延到周围适宜的土地上。

\subsubsection{口菇}

口菇的获取难度被设计为\textbf{困难},其特征如下:
\begin{table}[H]
    \centering
    \caption{口菇特征}
    \label{table:口菇特征}
    \setlength{\tabcolsep}{4mm}
    \begin{tabular}{c|cc|cc}
        \toprule
        \textbf{模型特征}   & 模型高度 & 0.5      & 模型形状 & 几何     \\
        \midrule
        \textbf{单方块特征} & 生长方块 & 灰化土/菌丝 & 替换方块 & 空气     \\
        \midrule
        \textbf{散植特征}   & 生成几率 & 0.2   & 生成次数 & 1        \\
        \midrule
        \textbf{特征规则}   & 生成地   & 地表   & 生态     & 寒带森林/蘑菇岛 \\
        \bottomrule
    \end{tabular}
\end{table}


口菇的种植难度被设计为\textbf{中等},其种植信息如下:

\begin{table}[H]
    \centering
    \caption{口菇种植信息}
    \label{table:口菇种植信息}
    \setlength{\tabcolsep}{4mm}
    \begin{tabular}{c|cc|cc}
        \toprule
                                           & \textbf{属性} & \textbf{说明} & \textbf{属性} & \textbf{说明} \\
        \midrule
        \multirow{4}{*}{\textbf{种植条件}} & 土地          & 灰化土/菌丝          & 生态(温度)    & 温带,寒带    \\
                                           & 生态(降雨)    & 中,低        & 生态(海拔)    & 中,中高            \\
                                           & 生态(土地)    & 草地          & 生态(特殊)    & -             \\
                                           & 特殊          & 菌类           \\
        \midrule
        \multirow{2}{*}{\textbf{生长条件}} & 光照          & [6,12]        & 海拔          & [64,192]      \\
                                           & 天气          & 无            & 特殊          & -             \\
        \midrule
        \textbf{生长速度}                  & 速度          & 快(6tick)     & 阶段tick      & [2,2,2]       \\
        \midrule
        \multirow{3}{*}{\textbf{收获}}     & 收获次数      & 1             & 掉落物        & 口菇:2-3      \\
                                           & 饥饿度        & 2             & 营养值        & 0.4           \\
                                           & 种子          & 果实即为种子  & 生吃          & 可(80\% 中毒 10s)   \\
        \bottomrule
    \end{tabular}
\end{table}

\subsubsection{香菇}

香菇的获取难度被设计为\textbf{困难},其特征如下:
\begin{table}[H]
    \centering
    \caption{香菇特征}
    \label{table:香菇特征}
    \setlength{\tabcolsep}{4mm}
    \begin{tabular}{c|cc|cc}
        \toprule
        \textbf{模型特征}   & 模型高度 & 0.5      & 模型形状 & 几何     \\
        \midrule
        \textbf{单方块特征} & 生长方块 & 灰化土/菌丝/草地 & 替换方块 & 空气     \\
        \midrule
        \textbf{散植特征}   & 生成几率 & 0.1   & 生成次数 & 3        \\
        \midrule
        \textbf{特征规则}   & 生成地   & 地表   & 生态     & 温带森林/蘑菇岛 \\
        \bottomrule
    \end{tabular}
\end{table}


香菇的种植难度被设计为\textbf{困难},其种植信息如下:

\begin{table}[H]
    \centering
    \caption{香菇种植信息}
    \label{table:香菇种植信息}
    \setlength{\tabcolsep}{4mm}
    \begin{tabular}{c|cc|cc}
        \toprule
                                           & \textbf{属性} & \textbf{说明} & \textbf{属性} & \textbf{说明} \\
        \midrule
        \multirow{4}{*}{\textbf{种植条件}} & 土地          & 朽木          & 生态(温度)    & 温带    \\
                                           & 生态(降雨)    & 中,低        & 生态(海拔)    & 中            \\
                                           & 生态(土地)    & 草地          & 生态(特殊)    & -             \\
                                           & 特殊          & 菌类           \\
        \midrule
        \multirow{2}{*}{\textbf{生长条件}} & 光照          & [6,12]        & 海拔          & [64,128]      \\
                                           & 天气          & 无            & 特殊          & -             \\
        \midrule
        \textbf{生长速度}                  & 速度          & 较快(9tick)     & 阶段tick      & [3,3,3]       \\
        \midrule
        \multirow{3}{*}{\textbf{收获}}     & 收获次数      & 1             & 掉落物        & 香菇:2-4      \\
                                           & 饥饿度        & 2             & 营养值        & 0.4           \\
                                           & 种子          & 果实即为种子  & 生吃          & 可(40\% 中毒 10s)   \\
        \bottomrule
    \end{tabular}
\end{table}

\subsubsection{金针菇}

金针菇的获取难度被设计为\textbf{中等},其特征如下:
\begin{table}[H]
    \centering
    \caption{金针菇特征}
    \label{table:金针菇特征}
    \setlength{\tabcolsep}{4mm}
    \begin{tabular}{c|cc|cc}
        \toprule
        \textbf{模型特征}   & 模型高度 & 1      & 模型形状 & 井字形     \\
        \midrule
        \textbf{单方块特征} & 生长方块 & 灰化土/菌丝/草地 & 替换方块 & 空气     \\
        \midrule
        \textbf{散植特征}   & 生成几率 & 0.1   & 生成次数 & 1        \\
        \midrule
        \textbf{特征规则}   & 生成地   & 地表   & 生态     & 温带,寒带森林/蘑菇岛 \\
        \bottomrule
    \end{tabular}
\end{table}


金针菇的种植难度被设计为\textbf{中等},其种植信息如下:

\begin{table}[H]
    \centering
    \caption{金针菇种植信息}
    \label{table:金针菇种植信息}
    \setlength{\tabcolsep}{4mm}
    \begin{tabular}{c|cc|cc}
        \toprule
                                           & \textbf{属性} & \textbf{说明} & \textbf{属性} & \textbf{说明} \\
        \midrule
        \multirow{4}{*}{\textbf{种植条件}} & 土地          &  灰化土/菌丝         & 生态(温度)    & 温带/寒带    \\
                                           & 生态(降雨)    & 中,低        & 生态(海拔)    & 中            \\
                                           & 生态(土地)    & 草地          & 生态(特殊)    & -             \\
                                           & 特殊          & 菌类           \\
        \midrule
        \multirow{2}{*}{\textbf{生长条件}} & 光照          & [6,12]        & 海拔          & [64,192]      \\
                                           & 天气          & 无            & 特殊          & -             \\
        \midrule
        \textbf{生长速度}                  & 速度          & 较快(8tick)     & 阶段tick      & [2,3,3]       \\
        \midrule
        \multirow{3}{*}{\textbf{收获}}     & 收获次数      & 1             & 掉落物        & 金针菇:3-4      \\
                                           & 饥饿度        & 2             & 营养值        & 0.4           \\
                                           & 种子          & 果实即为种子  & 生吃          & 可(50\% 中毒 10s)   \\
        \bottomrule
    \end{tabular}
\end{table}

\subsubsection{松茸}

松茸的获取难度被设计为\textbf{极难},其特征如下:
\begin{table}[H]
    \centering
    \caption{松茸特征}
    \label{table:松茸特征}
    \setlength{\tabcolsep}{4mm}
    \begin{tabular}{c|cc|cc}
        \toprule
        \textbf{模型特征}   & 模型高度 & 0.5      & 模型形状 & 几何     \\
        \midrule
        \textbf{单方块特征} & 生长方块 & 灰化土/菌丝 & 替换方块 & 空气     \\
        \midrule
        \textbf{散植特征}   & 生成几率 & 0.05   & 生成次数 & 1        \\
        \midrule
        \textbf{特征规则}   & 生成地   & 地表   & 生态     & 寒带森林/蘑菇岛 \\
        \bottomrule
    \end{tabular}
\end{table}


松茸的种植难度被设计为\textbf{极难},其种植信息如下:

\begin{table}[H]
    \centering
    \caption{松茸种植信息}
    \label{table:松茸种植信息}
    \setlength{\tabcolsep}{4mm}
    \begin{tabular}{c|cc|cc}
        \toprule
                                           & \textbf{属性} & \textbf{说明} & \textbf{属性} & \textbf{说明} \\
        \midrule
        \multirow{4}{*}{\textbf{种植条件}} & 土地          & 灰化土/菌丝    & 生态(温度)    & 亚寒带,寒带    \\
                                           & 生态(降雨)    & 中,低        & 生态(海拔)    & 中高,高            \\
                                           & 生态(土地)    & 草地          & 生态(特殊)    & -             \\
                                           & 特殊          & 菌类,无法蔓延           \\
        \midrule
        \multirow{2}{*}{\textbf{生长条件}} & 光照          & [6,12]        & 海拔          & [128,256]      \\
                                           & 天气          & 无            & 特殊          & -             \\
        \midrule
        \textbf{生长速度}                  & 速度          & 极慢(39tick)     & 阶段tick      & [13,13,13]       \\
        \midrule
        \multirow{3}{*}{\textbf{收获}}     & 收获次数      & 1             & 掉落物        & 松茸:1-5      \\
                                           & 饥饿度        & 3             & 营养值        & 0.4           \\
                                           & 种子          & 果实即为种子  & 生吃          & 可   \\
        \bottomrule
    \end{tabular}
\end{table}

\subsubsection{松露}

松露的获取难度被设计为\textbf{极难},其特征如下:
\begin{table}[H]
    \centering
    \caption{松露特征}
    \label{table:松露特征}
    \setlength{\tabcolsep}{4mm}
    \begin{tabular}{c|cc|cc}
        \toprule
        \textbf{模型特征}   & 模型高度 & 0.5      & 模型形状 & 几何     \\
        \midrule
        \textbf{单方块特征} & 生长方块 & 灰化土/菌丝 & 替换方块 & 空气     \\
        \midrule
        \textbf{散植特征}   & 生成几率 & 0.1   & 生成次数 & 1        \\
        \midrule
        \textbf{特征规则}   & 生成地   & 地表   & 生态     & 寒带森林/蘑菇岛 \\
        \bottomrule
    \end{tabular}
\end{table}


松露的种植难度被设计为\textbf{传奇},也即不可种植:

\begin{table}[H]
    \centering
    \caption{松露种植信息}
    \label{table:松露种植信息}
    \setlength{\tabcolsep}{4mm}
    \begin{tabular}{c|cc|cc}
        \toprule
                                           & \textbf{属性} & \textbf{说明} & \textbf{属性} & \textbf{说明} \\
        \midrule
        \multirow{3}{*}{\textbf{收获}}     & 收获次数      & 1             & 掉落物        & 松露:1-3      \\
                                           & 饥饿度        & 0             & 营养值        & 0.4           \\
                                           & 种子          & 无  & 生吃          & 不可   \\
        \bottomrule
    \end{tabular}
\end{table}

\subsubsection{黑松露}

黑松露的获取难度被设计为\textbf{传奇},其特征如下:
\begin{table}[H]
    \centering
    \caption{黑松露特征}
    \label{table:黑松露特征}
    \setlength{\tabcolsep}{4mm}
    \begin{tabular}{c|cc|cc}
        \toprule
        \textbf{模型特征}   & 模型高度 & 0.5      & 模型形状 & 几何     \\
        \midrule
        \textbf{单方块特征} & 生长方块 & 灰化土/菌丝 & 替换方块 & 空气     \\
        \midrule
        \textbf{散植特征}   & 生成几率 & 0.02   & 生成次数 & 1        \\
        \midrule
        \textbf{特征规则}   & 生成地   & 地表   & 生态     & 寒带森林 \\
        \bottomrule
    \end{tabular}
\end{table}


黑松露的种植难度被设计为\textbf{传奇},也即不可种植:

\begin{table}[H]
    \centering
    \caption{黑松露种植信息}
    \label{table:黑松露种植信息}
    \setlength{\tabcolsep}{4mm}
    \begin{tabular}{c|cc|cc}
        \toprule
                                           & \textbf{属性} & \textbf{说明} & \textbf{属性} & \textbf{说明} \\
        \midrule
        \multirow{3}{*}{\textbf{收获}}     & 收获次数      & 1             & 掉落物        & 黑松露:1      \\
                                           & 饥饿度        & 0             & 营养值        & 0.4           \\
                                           & 种子          & 无  & 生吃          & 不可   \\
        \bottomrule
    \end{tabular}
\end{table}

\subsection{水果}

水果的特点: 可直接食用,可做成饮料,可提供 buff。

\subsubsection{香蕉}

香蕉的获取难度被设计为\textbf{困难},其特征如下:
\begin{table}[H]
    \centering
    \caption{香蕉特征}
    \label{table:香蕉特征}
    \setlength{\tabcolsep}{4mm}
    \begin{tabular}{c|cc|cc}
        \toprule
        \textbf{模型特征}   & 模型高度 & 3      & 模型形状 & X字形 \\
        \midrule
        \textbf{单方块特征} & 生长方块 & 草方块 & 替换方块 & 空气  \\
        \midrule
        \textbf{散植特征}   & 生成几率 & 0.1    & 生成次数 & 1     \\
        \midrule
        \textbf{特征规则}   & 生成地   & 地表   & 生态     & 丛林  \\
        \bottomrule
    \end{tabular}
\end{table}


香蕉的种植难度被设计为\textbf{困难},其种植信息如下:

\begin{table}[H]
    \centering
    \caption{香蕉种植信息}
    \label{table:香蕉种植信息}
    \setlength{\tabcolsep}{4mm}
    \begin{tabular}{c|cc|cc}
        \toprule
                                           & \textbf{属性} & \textbf{说明} & \textbf{属性} & \textbf{说明} \\
        \midrule
        \multirow{4}{*}{\textbf{种植条件}} & 土地          & 农田          & 生态(温度)    & 热带          \\
                                           & 生态(降雨)    & 中,高        & 生态(海拔)    & 中            \\
                                           & 生态(土地)    & 草地          & 生态(特殊)    & -             \\
                                           & 特殊          & -                                             \\
        \midrule
        \multirow{2}{*}{\textbf{生长条件}} & 光照          & [9,15]        & 海拔          & [64,128]      \\
                                           & 天气          & -             & 特殊          & -             \\
        \midrule
        \textbf{生长速度}                  & 速度          & 慢(17tick)    & 阶段tick      & [3,3,4,3,4]   \\
        \midrule
        \multirow{3}{*}{\textbf{收获}}     & 收获次数      & 3             & 掉落物        & 香蕉:4-7      \\
                                           & 饥饿度        & 2             & 营养值        & 1.2           \\
                                           & 种子          & 果实即种子    & 生吃          & 可            \\
        \bottomrule
    \end{tabular}
\end{table}

\subsection{配料}

配料的特点: 不提供饱食度。

\subsubsection{香草}

香草的获取难度被设计为\textbf{简单},其特征如下:
\begin{table}[H]
    \centering
    \caption{香草特征}
    \label{table:香草特征}
    \setlength{\tabcolsep}{4mm}
    \begin{tabular}{c|cc|cc}
        \toprule
        \textbf{模型特征}   & 模型高度 & 1      & 模型形状 & 井字形     \\
        \midrule
        \textbf{单方块特征} & 生长方块 & 草方块 & 替换方块 & 空气       \\
        \midrule
        \textbf{散植特征}   & 生成几率 & 0.08   & 生成次数 & 7          \\
        \midrule
        \textbf{特征规则}   & 生成地   & 地表   & 生态     & 森林,平原 \\
        \bottomrule
    \end{tabular}
\end{table}


香草的种植难度被设计为\textbf{简单},其种植信息如下:

\begin{table}[H]
    \centering
    \caption{香草种植信息}
    \label{table:香草种植信息}
    \setlength{\tabcolsep}{4mm}
    \begin{tabular}{c|cc|cc}
        \toprule
                                           & \textbf{属性} & \textbf{说明} & \textbf{属性} & \textbf{说明}    \\
        \midrule
        \multirow{4}{*}{\textbf{种植条件}} & 土地          & 农田,草方块  & 生态(温度)    & 温带             \\
                                           & 生态(降雨)    & 中            & 生态(海拔)    & 中               \\
                                           & 生态(土地)    & 草地          & 生态(特殊)    & -                \\
                                           & 特殊          & -                                                \\
        \midrule
        \multirow{2}{*}{\textbf{生长条件}} & 光照          & [9,15]        & 海拔          & [64,192]         \\
                                           & 天气          & 无            & 特殊          & -                \\
        \midrule
        \textbf{生长速度}                  & 速度          & 中等(11tick)  & 阶段tick      & [3,3,5]          \\
        \midrule
        \multirow{2}{*}{\textbf{收获}}     & 收获次数      & 1             & 掉落物        & 种子1-2,香草1-4 \\
                                           & 饥饿度        & 0             & 营养值        & 0                \\
                                           & 种子          & 果实种子分离  & 生吃          & 不可             \\
        \bottomrule
    \end{tabular}
\end{table}

\subsubsection{辣椒}

辣椒的获取难度被设计为\textbf{中等},其特征如下:
\begin{table}[H]
    \centering
    \caption{辣椒特征}
    \label{table:辣椒特征}
    \setlength{\tabcolsep}{4mm}
    \begin{tabular}{c|cc|cc}
        \toprule
        \textbf{模型特征}                  & 模型高度 & 1      & 模型形状 & 井字形 \\
        \midrule
        \textbf{单方块特征}                & 生长方块 & 草方块 & 替换方块 & 空气   \\
        \midrule
        \textbf{散植特征}                  & 生成几率 & 0.2    & 生成次数 & 3      \\
        \midrule
        \multirow{2}{*}{\textbf{特征规则}} & 生成地   & 地表   & 生态     & 森林   \\
                                           & 生成条件 & 树下                       \\
        \bottomrule
    \end{tabular}
\end{table}


辣椒的种植难度被设计为\textbf{简单},其种植信息如下:

\begin{table}[H]
    \centering
    \caption{辣椒种植信息}
    \label{table:辣椒种植信息}
    \setlength{\tabcolsep}{4mm}
    \begin{tabular}{c|cc|cc}
        \toprule
                                           & \textbf{属性} & \textbf{说明} & \textbf{属性} & \textbf{说明} \\
        \midrule
        \multirow{4}{*}{\textbf{种植条件}} & 土地          & 农田          & 生态(温度)    & 温带          \\
                                           & 生态(降雨)    & 中            & 生态(海拔)    & 中            \\
                                           & 生态(土地)    & 草地          & 生态(特殊)    & -             \\
                                           & 特殊          & -                                             \\
        \midrule
        \multirow{2}{*}{\textbf{生长条件}} & 光照          & [9,15]        & 海拔          & [64,128]      \\
                                           & 天气          & 无            & 特殊          & -             \\
        \midrule
        \textbf{生长速度}                  & 速度          & 快(7tick)     & 阶段tick      & [2,2,3]       \\
        \midrule
        \multirow{2}{*}{\textbf{收获}}     & 收获次数      & 1             & 掉落物        & 辣椒 2-5      \\
                                           & 饥饿度        & 0             & 营养值        & 0             \\
                                           & 种子          & 果实即为种子  & 生吃          & 可(反胃15s)   \\
        \bottomrule
    \end{tabular}
\end{table}

\subsubsection{油菜花}

油菜花的获取难度被设计为\textbf{简单},其特征如下:
\begin{table}[H]
    \centering
    \caption{油菜花特征}
    \label{table:油菜花特征}
    \setlength{\tabcolsep}{4mm}
    \begin{tabular}{c|cc|cc}
        \toprule
        \textbf{模型特征}                  & 模型高度 & 1      & 模型形状 & 井字形 \\
        \midrule
        \textbf{单方块特征}                & 生长方块 & 草方块 & 替换方块 & 空气   \\
        \midrule
        \textbf{散植特征}                  & 生成几率 & 0.2    & 生成次数 & 5      \\
        \midrule
        \multirow{2}{*}{\textbf{特征规则}} & 生成地   & 地表   & 生态     & 草原   \\
                                           & 生成条件 & -                       \\
        \bottomrule
    \end{tabular}
\end{table}


油菜花的种植难度被设计为\textbf{简单},其种植信息如下:

\begin{table}[H]
    \centering
    \caption{油菜花种植信息}
    \label{table:油菜花种植信息}
    \setlength{\tabcolsep}{4mm}
    \begin{tabular}{c|cc|cc}
        \toprule
                                           & \textbf{属性} & \textbf{说明} & \textbf{属性} & \textbf{说明} \\
        \midrule
        \multirow{4}{*}{\textbf{种植条件}} & 土地          & 农田,草地          & 生态(温度)    & 温带,亚寒带,寒带  \\
                                           & 生态(降雨)    & 中,高            & 生态(海拔)    & 中,高            \\
                                           & 生态(土地)    & 草地          & 生态(特殊)    & -             \\
                                           & 特殊          & -                                             \\
        \midrule
        \multirow{2}{*}{\textbf{生长条件}} & 光照          & [9,15]        & 海拔          & [64,256]      \\
                                           & 天气          & 无            & 特殊          & -             \\
        \midrule
        \textbf{生长速度}                  & 速度          & 快(7tick)     & 阶段tick      & [2,2,3]       \\
        \midrule
        \multirow{2}{*}{\textbf{收获}}     & 收获次数      & 1             & 掉落物        & 油菜花 2-5      \\
                                           & 饥饿度        & 0             & 营养值        & 0             \\
                                           & 种子          & 果实即为种子  & 生吃          & 不可   \\
        \bottomrule
    \end{tabular}
\end{table}


\newpage