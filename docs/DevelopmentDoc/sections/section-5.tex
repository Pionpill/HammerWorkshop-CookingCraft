\section{植物系统}

原版植物生长仅需要土地,光照等简单条件,且仅含一次性收获(小麦)与藤曼生长(南瓜)两种方式,较为简单。模组的植物通过脚本控制,添加了植物种植条件:生态,攀爬。增添了可多次收获植物,树木...以丰富玩法。

\subsection{粮食}

粮食的特点: 多产,提供大量饱食度,无 buff。

\subsubsection{水稻}

\subsubsection{玉米}

\subsection{蔬菜}

蔬菜的特点: 中等产量,作为辅助原材料,一般无 buff。

\subsubsection{番茄}

\subsubsection{洋葱}

\subsection{水果}

水果的特点: 可直接食用,可做成饮料,可提供 buff。

\subsection{配料}

配料的特点: 不提供饱食度。

\subsubsection{香草}

香草的获取难度被设计为\textbf{简单},其特征如下:
\begin{table}[H]
    \centering
    \caption{香草特征}
    \label{table:香草特征}
    \setlength{\tabcolsep}{4mm}
    \begin{tabular}{c|cc|cc}
        \toprule
        \textbf{单方块特征} & 生长方块 & 草方块 & 替换方块 & 空气 \\
        \midrule
        \textbf{散植特征} & 生成几率 & 0.1 & 生成次数 & 7 \\
        \midrule
        \textbf{特征规则} & 生成地 & 地表 & 生态 & 森林,平原 \\
        \bottomrule
    \end{tabular}
\end{table}


香草的种植难度被设计为\textbf{简单},其种植信息如下:

\begin{table}[H]
    \centering
    \caption{香草种植信息}
    \label{table:香草种植信息}
    \setlength{\tabcolsep}{4mm}
    \begin{tabular}{c|cc|cc}
        \toprule
                                           & \textbf{属性} & \textbf{说明} & \textbf{属性} & \textbf{说明}        \\
        \midrule
        \textbf{种植条件}                  & 土地          & 农田,草方块    & 生态          & 温带平原,森林,水域 \\
        \midrule
        \multirow{2}{*}{\textbf{生长条件}} & 光照          & [9,15]        & 海拔          & [64,192]               \\
                                           & 天气          & 无            & 特殊          & -             \\
        \midrule
        \textbf{生长速度}                  & 速度          & 中等(11tick)    & 阶段tick      & [3,3,5]              \\
        \midrule
        \multirow{2}{*}{\textbf{收获}}     & 收获次数      & 1             & 掉落物        & 种子1-2,香草1-4     \\
                                           & 饥饿度        & 0             & 营养值        & 0                    \\
        \bottomrule
    \end{tabular}
\end{table}

\subsubsection{辣椒}

辣椒的获取难度被设计为\textbf{中等},其特征如下:
\begin{table}[H]
    \centering
    \caption{辣椒特征}
    \label{table:辣椒特征}
    \setlength{\tabcolsep}{4mm}
    \begin{tabular}{c|cc|cc}
        \toprule
        \textbf{单方块特征} & 生长方块 & 草方块 & 替换方块 & 空气 \\
        \midrule
        \textbf{散植特征} & 生成几率 & 0.2 & 生成次数 & 3 \\
        \midrule
        \multirow{2}{*}{\textbf{特征规则}} & 生成地 & 地表 & 生态 & 森林 \\
                        & 生成条件 & 树下  \\
        \bottomrule
    \end{tabular}
\end{table}


辣椒的种植难度被设计为\textbf{简单},其种植信息如下:

\begin{table}[H]
    \centering
    \caption{辣椒种植信息}
    \label{table:辣椒种植信息}
    \setlength{\tabcolsep}{4mm}
    \begin{tabular}{c|cc|cc}
        \toprule
                                           & \textbf{属性} & \textbf{说明} & \textbf{属性} & \textbf{说明}        \\
        \midrule
        \textbf{种植条件}                  & 土地          & 农田,草方块    & 生态          & 温带平原,森林 \\
        \midrule
        \multirow{2}{*}{\textbf{生长条件}} & 光照          & [9,15]        & 海拔          & [64,128]               \\
                                           & 天气          & 无            & 特殊          & -             \\
        \midrule
        \textbf{生长速度}                  & 速度          & 快(7tick)    & 阶段tick      & [2,2,3]              \\
        \midrule
        \multirow{2}{*}{\textbf{收获}}     & 收获次数      & 1             & 掉落物        & 辣椒 2-5     \\
                                           & 饥饿度        & 0             & 营养值        & 0                    \\
        \bottomrule
    \end{tabular}
\end{table}

\newpage