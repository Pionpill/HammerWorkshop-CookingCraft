\section{植物系统}

原版植物生长仅需要土地,光照等简单条件,且仅含一次性收获(小麦)与藤曼生长(南瓜)两种方式,较为简单。模组的植物通过脚本控制,添加了植物种植条件:生态,攀爬。增添了可多次收获植物,树木...以丰富玩法。

我们将植物按作用分为三类: 粮食,蔬菜,水果,调料。植物的类被决定了植物果实在合成食物的过程中承担的角色(主要原材料,调料等)。同时不同的类别有其特点,比如粮食能大量生产,提供大量饱食度。

其次,按照现实世界中的植物类型,我们将植物分为草本,禾本,藤蔓,木本,菌类,特殊。其在 MC 中的特性如下\footnote{此为大部分植物遵守的准则,也有例外,如香蕉这一大型草本植物,可收获3次}:

\begin{table}[H]
    \centering
    \caption{植物类别特性}
    \label{table:植物类别特性}
    \setlength{\tabcolsep}{4mm}
    \begin{tabular}{c|ccccc}
        \toprule
        \textbf{植物类型} & \textbf{草本} & \textbf{禾本} & \textbf{藤蔓} & \textbf{木本} & \textbf{菌类} \\
        \midrule
        生长土地          & 草地,田地    & 田地          & 田地          & 草地          & 灰化土        \\
        生长时间          & 快            & 中等          & 中等          & 慢            & 中等          \\
        收获次数          & 1             & 1-3           & 1-5           & 特殊          & 特殊          \\
        生长方式          & 直接种植      & 直接种植      & 攀藤          & 种植生长      & 直接种植      \\
        \bottomrule
    \end{tabular}
\end{table}

此外,针对植物的获取与种植条件,设置难度如下。

\begin{table}[H]
    \centering
    \caption{植物获取}
    \label{table:植物获取}
    \setlength{\tabcolsep}{4mm}
    \begin{tabular}{c|ccc}
        \toprule
        \textbf{获取难度} & \textbf{表示}                                  & \textbf{说明}                   & \textbf{例子} \\
        \midrule
        简单              & $\bigstar$                                     & 世界大概率遇见                  & 香草          \\
        中等              & $\bigstar\bigstar$                             & 世界中等概率遇见                & 辣椒          \\
        困难              & $\bigstar \bigstar \bigstar$                   & 世界小概率遇见,怪物掉落        & 水稻          \\
        极难              & $\bigstar \bigstar \bigstar \bigstar$          & 条件极为严苛或通过购买/宝箱获得 & 雪莲          \\
        传奇              & $\bigstar \bigstar \bigstar \bigstar \bigstar$ & 极其特殊的获取方式              & 暂无          \\
        \midrule
        \textbf{种植难度} & \textbf{表示}                                  & \textbf{说明}                   & \textbf{例子} \\
        \midrule
        简单              & $\bigstar$                                     & 生态等条件宽裕                  & 香草          \\
        中等              & $\bigstar\bigstar$                             & 需要特殊的条件,但容易达到      & 水稻          \\
        困难              & $\bigstar \bigstar \bigstar$                   & 种植生态较少                    & 香蕉          \\
        极难              & $\bigstar \bigstar \bigstar \bigstar$          & 严苛的种植条件                  & 雪莲          \\
        传奇              & $\bigstar \bigstar \bigstar \bigstar \bigstar$ & 无法种植或极难达到              & 暂无          \\
        \bottomrule
    \end{tabular}
\end{table}

\subsection{粮食}

粮食的特点: 多产,提供大量饱食度,无 buff。

\subsubsection{水稻}

水稻的获取难度被设计为\textbf{困难},其特征如下:
\begin{table}[H]
    \centering
    \caption{水稻特征}
    \label{table:水稻特征}
    \setlength{\tabcolsep}{4mm}
    \begin{tabular}{c|cc|cc}
        \toprule
        \textbf{模型特征}                  & 模型高度 & 1      & 模型形状 & 井字形 \\
        \midrule
        \textbf{单方块特征}                & 生长方块 & 草方块 & 替换方块 & 空气   \\
        \midrule
        \textbf{散植特征}                  & 生成几率 & 0.8    & 生成次数 & 3      \\
        \midrule
        \multirow{2}{*}{\textbf{特征规则}} & 生成地   & 地表   & 生态     & 水系   \\
                                           & 生成条件 & 河边                       \\
        \bottomrule
    \end{tabular}
\end{table}


水稻的种植难度被设计为\textbf{中等},其种植信息如下:

\begin{table}[H]
    \centering
    \caption{水稻种植信息}
    \label{table:水稻种植信息}
    \setlength{\tabcolsep}{4mm}
    \begin{tabular}{c|cc|cc}
        \toprule
                                           & \textbf{属性} & \textbf{说明}   & \textbf{属性} & \textbf{说明} \\
        \midrule
        \multirow{4}{*}{\textbf{种植条件}} & 土地          & 农田            & 生态(温度)    & 温带,热带    \\
                                           & 生态(降雨)    & 中,高          & 生态(海拔)    & 中            \\
                                           & 生态(土地)    & 草地            & 生态(特殊)    & 无            \\
                                           & 特殊          & 附近 3 格内有水                                 \\
        \midrule
        \multirow{2}{*}{\textbf{生长条件}} & 光照          & [11,15]         & 海拔          & [64,128]      \\
                                           & 天气          & 无              & 特殊          & 下雨发芽      \\
        \midrule
        \textbf{生长速度}                  & 速度          & 快(7tick)       & 阶段tick      & [2,2,3]       \\
        \midrule
        \multirow{3}{*}{\textbf{收获}}     & 收获次数      & 1               & 掉落物        & 水稻:3-7      \\
                                           & 饥饿度        & 2               & 营养值        & 0.4           \\
                                           & 种子          & 果实分离        & 生吃          & 不可          \\
        \bottomrule
    \end{tabular}
\end{table}

\subsubsection{玉米}

玉米的获取难度被设计为\textbf{中等},其特征如下:
\begin{table}[H]
    \centering
    \caption{玉米特征}
    \label{table:玉米特征}
    \setlength{\tabcolsep}{4mm}
    \begin{tabular}{c|cc|cc}
        \toprule
        \textbf{模型特征}   & 模型高度 & 2      & 模型形状 & X字  \\
        \midrule
        \textbf{单方块特征} & 生长方块 & 草方块 & 替换方块 & 空气 \\
        \midrule
        \textbf{散植特征}   & 生成几率 & 0.05   & 生成次数 & 1    \\
        \midrule
        \textbf{特征规则}   & 生成地   & 地表   & 生态     & 平原 \\
        \bottomrule
    \end{tabular}
\end{table}


玉米的种植难度被设计为\textbf{简单},其种植信息如下:

\begin{table}[H]
    \centering
    \caption{玉米种植信息}
    \label{table:玉米种植信息}
    \setlength{\tabcolsep}{4mm}
    \begin{tabular}{c|cc|cc}
        \toprule
                                           & \textbf{属性} & \textbf{说明} & \textbf{属性} & \textbf{说明} \\
        \midrule
        \multirow{4}{*}{\textbf{种植条件}} & 土地          & 农田          & 生态(温度)    & 温带,热带    \\
                                           & 生态(降雨)    & 中            & 生态(海拔)    & 中            \\
                                           & 生态(土地)    & 草地          & 生态(特殊)    & 针叶林        \\
                                           & 特殊          & -                                             \\
        \midrule
        \multirow{2}{*}{\textbf{生长条件}} & 光照          & [9,15]        & 海拔          & [64,192]      \\
                                           & 天气          & 无            & 特殊          & -             \\
        \midrule
        \textbf{生长速度}                  & 速度          & 较慢(14tick)  & 阶段tick      & [3,3,3,5]     \\
        \midrule
        \multirow{2}{*}{\textbf{收获}}     & 收获次数      & 3             & 掉落物        & 玉米:2-3      \\
                                           & 饥饿度        & 3             & 营养值        & 0.6           \\
                                           & 种子          & 果实分离      & 生吃          & 不可          \\
        \bottomrule
    \end{tabular}
\end{table}

\subsection{蔬菜}

蔬菜的特点: 中等产量,作为辅助原材料,一般无 buff。

\subsubsection{番茄}

番茄的获取难度被设计为\textbf{困难},其特征如下:
\begin{table}[H]
    \centering
    \caption{番茄特征}
    \label{table:番茄特征}
    \setlength{\tabcolsep}{4mm}
    \begin{tabular}{c|cc|cc}
        \toprule
        \textbf{模型特征}   & 模型高度 & 1      & 模型形状 & 井字 \\
        \midrule
        \textbf{单方块特征} & 生长方块 & 草方块 & 替换方块 & 空气 \\
        \midrule
        \textbf{散植特征}   & 生成几率 & -      & 生成次数 & -    \\
        \midrule
        \textbf{特征规则}   & 生成地   & 地表   & 生态     & 平原 \\
        \bottomrule
    \end{tabular}
\end{table}


番茄的种植难度被设计为\textbf{简单},其种植信息如下:

\begin{table}[H]
    \centering
    \caption{番茄种植信息}
    \label{table:番茄种植信息}
    \setlength{\tabcolsep}{4mm}
    \begin{tabular}{c|cc|cc}
        \toprule
                                           & \textbf{属性} & \textbf{说明} & \textbf{属性} & \textbf{说明} \\
        \midrule
        \multirow{4}{*}{\textbf{种植条件}} & 土地          & 篱笆          & 生态(温度)    & 温带,热带    \\
                                           & 生态(降雨)    & 中            & 生态(海拔)    & 中            \\
                                           & 生态(土地)    & 草地          & 生态(特殊)    & -             \\
                                           & 特殊          & -                                             \\
        \midrule
        \multirow{2}{*}{\textbf{生长条件}} & 光照          & [6,15]        & 海拔          & [64,128]      \\
                                           & 天气          & 无            & 特殊          & -             \\
        \midrule
        \textbf{生长速度}                  & 速度          & 中等(11tick)  & 阶段tick      & [3,4,4]       \\
        \midrule
        \multirow{3}{*}{\textbf{收获}}     & 收获次数      & 2             & 掉落物        & 番茄:2-5      \\
                                           & 饥饿度        & 2             & 营养值        & 0.6           \\
                                           & 种子          & 果实即为种子  & 生吃          & 可            \\
        \bottomrule
    \end{tabular}
\end{table}

\subsubsection{洋葱}

洋葱的获取难度被设计为\textbf{中等},其特征如下:
\begin{table}[H]
    \centering
    \caption{洋葱特征}
    \label{table:洋葱特征}
    \setlength{\tabcolsep}{4mm}
    \begin{tabular}{c|cc|cc}
        \toprule
        \textbf{模型特征}   & 模型高度 & 1      & 模型形状 & 井字     \\
        \midrule
        \textbf{单方块特征} & 生长方块 & 草方块 & 替换方块 & 空气     \\
        \midrule
        \textbf{散植特征}   & 生成几率 & 0.05   & 生成次数 & 3        \\
        \midrule
        \textbf{特征规则}   & 生成地   & 地表   & 生态     & 寒带森林 \\
        \bottomrule
    \end{tabular}
\end{table}


洋葱的种植难度被设计为\textbf{简单},其种植信息如下:

\begin{table}[H]
    \centering
    \caption{洋葱种植信息}
    \label{table:洋葱种植信息}
    \setlength{\tabcolsep}{4mm}
    \begin{tabular}{c|cc|cc}
        \toprule
                                           & \textbf{属性} & \textbf{说明} & \textbf{属性} & \textbf{说明} \\
        \midrule
        \multirow{4}{*}{\textbf{种植条件}} & 土地          & 农田          & 生态(温度)    & 温带,寒带    \\
                                           & 生态(降雨)    & 中,低        & 生态(海拔)    & 中            \\
                                           & 生态(土地)    & 草地          & 生态(特殊)    & -             \\
                                           & 特殊          & -                                             \\
        \midrule
        \multirow{2}{*}{\textbf{生长条件}} & 光照          & [9,15]        & 海拔          & [64,192]      \\
                                           & 天气          & 无            & 特殊          & -             \\
        \midrule
        \textbf{生长速度}                  & 速度          & 快(7tick)     & 阶段tick      & [2,2,3]       \\
        \midrule
        \multirow{3}{*}{\textbf{收获}}     & 收获次数      & 1             & 掉落物        & 洋葱:2-5      \\
                                           & 饥饿度        & 2             & 营养值        & 0.4           \\
                                           & 种子          & 果实即为种子  & 生吃          & 可(失明30s)   \\
        \bottomrule
    \end{tabular}
\end{table}

\subsection{水果}

水果的特点: 可直接食用,可做成饮料,可提供 buff。

\subsubsection{香蕉}

香蕉的获取难度被设计为\textbf{困难},其特征如下:
\begin{table}[H]
    \centering
    \caption{香蕉特征}
    \label{table:香蕉特征}
    \setlength{\tabcolsep}{4mm}
    \begin{tabular}{c|cc|cc}
        \toprule
        \textbf{模型特征}   & 模型高度 & 3      & 模型形状 & X字形 \\
        \midrule
        \textbf{单方块特征} & 生长方块 & 草方块 & 替换方块 & 空气  \\
        \midrule
        \textbf{散植特征}   & 生成几率 & 0.1    & 生成次数 & 1     \\
        \midrule
        \textbf{特征规则}   & 生成地   & 地表   & 生态     & 丛林  \\
        \bottomrule
    \end{tabular}
\end{table}


香蕉的种植难度被设计为\textbf{困难},其种植信息如下:

\begin{table}[H]
    \centering
    \caption{香蕉种植信息}
    \label{table:香蕉种植信息}
    \setlength{\tabcolsep}{4mm}
    \begin{tabular}{c|cc|cc}
        \toprule
                                           & \textbf{属性} & \textbf{说明} & \textbf{属性} & \textbf{说明} \\
        \midrule
        \multirow{4}{*}{\textbf{种植条件}} & 土地          & 农田          & 生态(温度)    & 热带          \\
                                           & 生态(降雨)    & 中,高        & 生态(海拔)    & 中            \\
                                           & 生态(土地)    & 草地          & 生态(特殊)    & -             \\
                                           & 特殊          & -                                             \\
        \midrule
        \multirow{2}{*}{\textbf{生长条件}} & 光照          & [9,15]        & 海拔          & [64,128]      \\
                                           & 天气          & -             & 特殊          & -             \\
        \midrule
        \textbf{生长速度}                  & 速度          & 慢(17tick)    & 阶段tick      & [3,3,4,3,4]   \\
        \midrule
        \multirow{3}{*}{\textbf{收获}}     & 收获次数      & 3             & 掉落物        & 香蕉:4-7      \\
                                           & 饥饿度        & 2             & 营养值        & 1.2           \\
                                           & 种子          & 果实即种子    & 生吃          & 可            \\
        \bottomrule
    \end{tabular}
\end{table}


\subsection{配料}

配料的特点: 不提供饱食度。

\subsubsection{香草}

香草的获取难度被设计为\textbf{简单},其特征如下:
\begin{table}[H]
    \centering
    \caption{香草特征}
    \label{table:香草特征}
    \setlength{\tabcolsep}{4mm}
    \begin{tabular}{c|cc|cc}
        \toprule
        \textbf{模型特征}   & 模型高度 & 1      & 模型形状 & 井字形     \\
        \midrule
        \textbf{单方块特征} & 生长方块 & 草方块 & 替换方块 & 空气       \\
        \midrule
        \textbf{散植特征}   & 生成几率 & 0.08   & 生成次数 & 7          \\
        \midrule
        \textbf{特征规则}   & 生成地   & 地表   & 生态     & 森林,平原 \\
        \bottomrule
    \end{tabular}
\end{table}


香草的种植难度被设计为\textbf{简单},其种植信息如下:

\begin{table}[H]
    \centering
    \caption{香草种植信息}
    \label{table:香草种植信息}
    \setlength{\tabcolsep}{4mm}
    \begin{tabular}{c|cc|cc}
        \toprule
                                           & \textbf{属性} & \textbf{说明} & \textbf{属性} & \textbf{说明}    \\
        \midrule
        \multirow{4}{*}{\textbf{种植条件}} & 土地          & 农田,草方块  & 生态(温度)    & 温带             \\
                                           & 生态(降雨)    & 中            & 生态(海拔)    & 中               \\
                                           & 生态(土地)    & 草地          & 生态(特殊)    & -                \\
                                           & 特殊          & -                                                \\
        \midrule
        \multirow{2}{*}{\textbf{生长条件}} & 光照          & [9,15]        & 海拔          & [64,192]         \\
                                           & 天气          & 无            & 特殊          & -                \\
        \midrule
        \textbf{生长速度}                  & 速度          & 中等(11tick)  & 阶段tick      & [3,3,5]          \\
        \midrule
        \multirow{2}{*}{\textbf{收获}}     & 收获次数      & 1             & 掉落物        & 种子1-2,香草1-4 \\
                                           & 饥饿度        & 0             & 营养值        & 0                \\
                                           & 种子          & 果实种子分离  & 生吃          & 不可             \\
        \bottomrule
    \end{tabular}
\end{table}

\subsubsection{辣椒}

辣椒的获取难度被设计为\textbf{中等},其特征如下:
\begin{table}[H]
    \centering
    \caption{辣椒特征}
    \label{table:辣椒特征}
    \setlength{\tabcolsep}{4mm}
    \begin{tabular}{c|cc|cc}
        \toprule
        \textbf{模型特征}                  & 模型高度 & 1      & 模型形状 & 井字形 \\
        \midrule
        \textbf{单方块特征}                & 生长方块 & 草方块 & 替换方块 & 空气   \\
        \midrule
        \textbf{散植特征}                  & 生成几率 & 0.2    & 生成次数 & 3      \\
        \midrule
        \multirow{2}{*}{\textbf{特征规则}} & 生成地   & 地表   & 生态     & 森林   \\
                                           & 生成条件 & 树下                       \\
        \bottomrule
    \end{tabular}
\end{table}


辣椒的种植难度被设计为\textbf{简单},其种植信息如下:

\begin{table}[H]
    \centering
    \caption{辣椒种植信息}
    \label{table:辣椒种植信息}
    \setlength{\tabcolsep}{4mm}
    \begin{tabular}{c|cc|cc}
        \toprule
                                           & \textbf{属性} & \textbf{说明} & \textbf{属性} & \textbf{说明} \\
        \midrule
        \multirow{4}{*}{\textbf{种植条件}} & 土地          & 农田          & 生态(温度)    & 温带          \\
                                           & 生态(降雨)    & 中            & 生态(海拔)    & 中            \\
                                           & 生态(土地)    & 草地          & 生态(特殊)    & -             \\
                                           & 特殊          & -                                             \\
        \midrule
        \multirow{2}{*}{\textbf{生长条件}} & 光照          & [9,15]        & 海拔          & [64,128]      \\
                                           & 天气          & 无            & 特殊          & -             \\
        \midrule
        \textbf{生长速度}                  & 速度          & 快(7tick)     & 阶段tick      & [2,2,3]       \\
        \midrule
        \multirow{2}{*}{\textbf{收获}}     & 收获次数      & 1             & 掉落物        & 辣椒 2-5      \\
                                           & 饥饿度        & 0             & 营养值        & 0             \\
                                           & 种子          & 果实即为种子  & 生吃          & 可(反胃15s)   \\
        \bottomrule
    \end{tabular}
\end{table}

\newpage