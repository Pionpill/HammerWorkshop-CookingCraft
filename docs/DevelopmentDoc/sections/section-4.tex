\section{工作台系统}

功能设计主要涉及到工作台相关的设计以及交互式方块设计,这里只对功能进行说明,详细的设计逻辑参考详细设计说明书。工作台大致可分为以下几类:
\begin{itemize}
    \item 工作台: UI 界面,保存部分 BlockEntity 数据,无燃料。
    \item 材料熔炉: UI 界面,保存所有 BlockEntity 数据,产物为中间产品。
    \item 产品熔炉: UI 界面,保存所有 BlockEntity 数据,产物为最终产品。
    \item 响应方块: 无 UI,需要保存 BlockEntity 数据,利用外观进行区分。
\end{itemize}

\subsection{工作台}

工作台,需要复杂的合成配方,一般不需要保持 BlockEntity 数据。

\subsubsection{厨务台}

厨务台 (cooking\_table) 为模组食物的主要加工区域,基本所有的食物原料都需要在此合成,再进入下一步加工。(例如苹果派需要在厨务台上合成为生苹果派,再放入烘焙炉)。

厨务台的合成配方如下:

\begin{figure}[H]
    \centering
    \begin{tikzpicture}
        \draw [recipe grid](0,0) grid (3,3);
        \node (1) at (0.5,2.5) {};
        \node (2) at (1.5,2.5) {};
        \node (3) at (2.5,2.5) {};
        \node (4) at (0.5,1.5)  {\includegraphics[width=0.8cm,height=0.8cm]{./images/origin/iron_ingot.png}};
        \node (5) at (1.5,1.5)  {\includegraphics[width=0.8cm,height=0.8cm]{./images/origin/oak_planks.png}};
        \node (6) at (2.5,1.5)  {};
        \node (7) at (0.5,0.5) {\includegraphics[width=0.8cm,height=0.8cm]{./images/origin/oak_planks.png}};
        \node (8) at (1.5,0.5)  {\includegraphics[width=0.8cm,height=0.8cm]{./images/origin/oak_planks.png}};
        \node (9) at (2.5,0.5) {};
        \draw [recipe grid] (6,1) grid (7,2);
        \node (10) at (6.5,1.5) {\includegraphics[width=0.8cm,height=0.8cm]{./images/mod/cooking_table_icon.png}};
        \begin{scope}
            \draw [-{Stealth},thick] (3.5,1.5) -- (5.5,1.5);
            \node (text) [font=\small] at (4.5,1.2) {工作台};
        \end{scope}
    \end{tikzpicture}
    \caption{面粉}
\end{figure}

厨务台和原版工作台类似,模组食物仅能在厨务台上合成。在此工作台上可以合成两种食物:
\begin{itemize}
    \item 生食(unfinished\_food): 需要进一步加工的食物,可以直接食用,但有负面buff
    \item 熟食(finished\_food): 可以直接使用的食物。
\end{itemize}

\begin{table}[H]
    \centering
    \caption{厨务台属性}
    \setlength{\tabcolsep}{4mm}
    \begin{tabular}{c|cc|c}
        \toprule
        \textbf{属性} & \textbf{说明} & \textbf{属性} & \textbf{说明} \\
        \midrule
        大小 & 1x1 & 作用 & 制作模组食物 \\
        $ER_{unfinished}$ & 0.4 & $ER_{finished}$ & 0.8 \\
        \bottomrule
    \end{tabular}
\end{table}

对生食进一步烹饪的过程中,需要将其携带的饥饿值乘 2 ,即按照熟食的饥饿值进一步加工。

其合成面板应具有以下结构:

\begin{figure}[H]
    \centering
    \begin{tikzpicture}
        \draw [recipe grid](0,0) grid (3,3);
        \draw [recipe grid](6,1) grid (7,2);
        \begin{scope}[yshift=-0.5cm]
            \draw [recipe grid](8,0) grid (9,4);
        \end{scope}
        \begin{scope}
            \draw [-{Stealth},thick] (3.5,1.5) -- (5.5,1.5);
            \node (text) [font=\small] at (4.5,1.2) {};
        \end{scope}
    \end{tikzpicture}
    \caption{厨务台 UI}
\end{figure}

\begin{itemize}
    \item 左侧: 九宫格为九个原材料槽
    \item 中间: 一个输出槽
    \item 右侧:四个固定物品的调料槽,依次为盐,糖,油,香料。
\end{itemize}

\subsubsection{屠宰台}

屠宰台 (butcher\_table) 负责对肉类食材进行加工。

屠宰台的合成配方如下:

\begin{figure}[H]
    \centering
    \begin{tikzpicture}
        \draw [recipe grid](0,0) grid (3,3);
        \node (1) at (0.5,2.5) {};
        \node (2) at (1.5,2.5) {};
        \node (3) at (2.5,2.5) {};
        \node (4) at (0.5,1.5)  {\includegraphics[width=0.8cm,height=0.8cm]{./images/origin/iron_ingot.png}};
        \node (5) at (1.5,1.5)  {};
        \node (6) at (2.5,1.5)  {};
        \node (7) at (0.5,0.5) {\includegraphics[width=0.8cm,height=0.8cm]{./images/origin/oak_log.png}};
        \node (8) at (1.5,0.5)  {};
        \node (9) at (2.5,0.5) {};
        \draw [recipe grid] (6,1) grid (7,2);
        \node (10) at (6.5,1.5) {\includegraphics[width=0.8cm,height=0.8cm]{./images/mod/butcher_block_icon.png}};
        \begin{scope}
            \draw [-{Stealth},thick] (3.5,1.5) -- (5.5,1.5);
            \node (text) [font=\small] at (4.5,1.2) {工作台};
        \end{scope}
    \end{tikzpicture}
    \caption{面粉}
\end{figure}


\begin{table}[H]
    \centering
    \caption{屠宰台属性}
    \setlength{\tabcolsep}{4mm}
    \begin{tabular}{c|cc|c}
        \toprule
        \textbf{属性} & \textbf{说明} & \textbf{属性} & \textbf{说明} \\
        \midrule
        大小 & 1x1 & 作用 & 加工肉类食物 \\
        $ER_{unfinished}$ & 0.4 & $ER_{finished}$ & 0.8 \\
        \bottomrule
    \end{tabular}
\end{table}

其合成面板应具有以下结构:

\begin{figure}[H]
    \centering
    \begin{tikzpicture}
        \draw [recipe grid](0,0) grid (3,3);
        \draw [recipe grid](6,1) grid (7,2);
        \begin{scope}[yshift=-0.5cm]
            \draw [recipe grid](8,0) grid (9,4);
        \end{scope}
        \begin{scope}
            \draw [-{Stealth},thick] (3.5,1.5) -- (5.5,1.5);
            \node (text) [font=\small] at (4.5,1.2) {};
        \end{scope}
    \end{tikzpicture}
    \caption{屠宰台 UI}
\end{figure}

\begin{itemize}
    \item 左侧: 九宫格为九个原材料槽
    \item 中间: 一个输出槽
    \item 右侧:四个固定物品的调料槽,依次为盐,糖,油,香料。
\end{itemize}

\subsection{材料熔炉}

\subsubsection{熔炉}

烘焙炉 (furnace) 提供火力,需要在熔炉上增加其他方块构成完整的熔炉。

烘焙炉的合成配方如下:

\begin{figure}[H]
    \centering
    \begin{tikzpicture}
        \draw [recipe grid](0,0) grid (3,3);
        \node (1) at (0.5,2.5) {};
        \node (2) at (1.5,2.5) {};
        \node (3) at (2.5,2.5) {};
        \node (4) at (0.5,1.5)  {\includegraphics[width=0.8cm,height=0.8cm]{./images/origin/clay_ball.png}};
        \node (5) at (1.5,1.5)  {\includegraphics[width=0.8cm,height=0.8cm]{./images/origin/clay_ball.png}};
        \node (6) at (2.5,1.5)  {};
        \node (7) at (0.5,0.5) {\includegraphics[width=0.8cm,height=0.8cm]{./images/origin/brick.png}};
        \node (8) at (1.5,0.5)  {\includegraphics[width=0.8cm,height=0.8cm]{./images/origin/brick.png}};
        \node (9) at (2.5,0.5) {};
        \draw [recipe grid] (6,1) grid (7,2);
        \node (10) at (6.5,1.5) {\includegraphics[width=0.8cm,height=0.8cm]{./images/mod/furnace_icon.png}};
        \begin{scope}
            \draw [-{Stealth},thick] (3.5,1.5) -- (5.5,1.5);
            \node (text) [font=\small] at (4.5,1.2) {工作台};
        \end{scope}
    \end{tikzpicture}
    \caption{面粉}
\end{figure}

\subsubsection{烘焙炉}

烘焙炉 (baking\_furnace) 用于将生食烘焙成熟。

烘焙炉的合成配方如下:

\begin{figure}[H]
    \centering
    \begin{tikzpicture}
        \draw [recipe grid](0,0) grid (3,3);
        \node (1) at (0.5,2.5) {};
        \node (2) at (1.5,2.5) {};
        \node (3) at (2.5,2.5) {};
        \node (4) at (0.5,1.5)  {\includegraphics[width=0.8cm,height=0.8cm]{./images/origin/brick.png}};
        \node (5) at (1.5,1.5)  {\includegraphics[width=0.8cm,height=0.8cm]{./images/origin/brick.png}};
        \node (6) at (2.5,1.5)  {\includegraphics[width=0.8cm,height=0.8cm]{./images/origin/brick.png}};
        \node (7) at (0.5,0.5) {\includegraphics[width=0.8cm,height=0.8cm]{./images/origin/brick.png}};
        \node (8) at (1.5,0.5)  {};
        \node (9) at (2.5,0.5) {\includegraphics[width=0.8cm,height=0.8cm]{./images/origin/brick.png}};
        \draw [recipe grid] (6,1) grid (7,2);
        \node (10) at (6.5,1.5) {\includegraphics[width=0.8cm,height=0.8cm]{./images/mod/baking_clay_icon.png}};
        \begin{scope}
            \draw [-{Stealth},thick] (3.5,1.5) -- (5.5,1.5);
            \node (text) [font=\small] at (4.5,1.2) {工作台};
        \end{scope}
    \end{tikzpicture}
    \caption{面粉}
\end{figure}

烘焙炉与原版熔炉功能相同,配方不同,在烘焙炉上可以将生的烘焙食物烤熟。

\begin{table}[H]
    \centering
    \caption{厨务台属性}
    \setlength{\tabcolsep}{4mm}
    \begin{tabular}{c|cc|c}
        \toprule
        \textbf{属性} & \textbf{说明} & \textbf{属性} & \textbf{说明} \\
        \midrule
        大小 & 2x2 & 作用 & 烘焙食物 \\
        饥饿度增量 & 2-5 & 经验 & 0.35 \\
        \bottomrule
    \end{tabular}
\end{table}

其熔炉面板应具有以下结构:

\begin{figure}[H]
    \centering
    \begin{tikzpicture}
        \draw [recipe grid](0,1) grid (1,2);
        \draw [recipe grid](4,0) grid (5,1);
        \draw [recipe grid](0,-1) grid (1,0);
        \begin{scope}
            \draw [-{Stealth},thick] (1.5,0.5) -- (3.5,0.5);
        \end{scope}
    \end{tikzpicture}
    \caption{烘焙炉 UI}
\end{figure}

\begin{itemize}
    \item 左上方: 原材料槽
    \item 左下方: 燃料槽
    \item 右侧:输出槽
\end{itemize}

\subsubsection{研磨机}

研磨机(mill) 对原材料(主要是谷物) 加工成食物原材料。

研磨机的合成配方如下:

\begin{figure}[H]
    \centering
    \begin{tikzpicture}
        \draw [recipe grid](0,0) grid (3,3);
        \node (1) at (0.5,2.5) {};
        \node (2) at (1.5,2.5) {};
        \node (3) at (2.5,2.5) {};
        \node (4) at (0.5,1.5)  {\includegraphics[width=0.8cm,height=0.8cm]{./images/origin/stone_bricks.png}};
        \node (5) at (1.5,1.5)  {\includegraphics[width=0.8cm,height=0.8cm]{./images/origin/stone_bricks.png}};
        \node (6) at (2.5,1.5)  {};
        \node (7) at (0.5,0.5) {\includegraphics[width=0.8cm,height=0.8cm]{./images/origin/oak_planks.png}};
        \node (8) at (1.5,0.5)  {\includegraphics[width=0.8cm,height=0.8cm]{./images/origin/oak_planks.png}};
        \node (9) at (2.5,0.5) {};
        \draw [recipe grid] (6,1) grid (7,2);
        \node (10) at (6.5,1.5) {\includegraphics[width=0.8cm,height=0.8cm]{./images/mod/mill_icon.png}};
        \begin{scope}
            \draw [-{Stealth},thick] (3.5,1.5) -- (5.5,1.5);
            \node (text) [font=\small] at (4.5,1.2) {工作台};
        \end{scope}
    \end{tikzpicture}
    \caption{面粉}
\end{figure}

研磨机类似与原版熔炉,原材料是金而不是燃料(有钱能使鬼推磨)。

\begin{table}[H]
    \centering
    \caption{研磨机属性}
    \setlength{\tabcolsep}{4mm}
    \begin{tabular}{c|cc|c}
        \toprule
        \textbf{属性} & \textbf{说明} & \textbf{属性} & \textbf{说明} \\
        \midrule
        大小 & 1x2 & 作用 & 加工谷物 \\
        饥饿度增量 & - & 经验 & 0.35 \\
        \bottomrule
    \end{tabular}
\end{table}

其熔炉面板应具有以下结构:

\begin{figure}[H]
    \centering
    \begin{tikzpicture}
        \draw [recipe grid](0,0) grid (1,1);
        \draw [recipe grid](4,0) grid (6,1);
        \draw [recipe grid](2,-2) grid (3,-1);
        \begin{scope}
            \draw [-{Stealth},thick] (1.5,0.5) -- (3.5,0.5);
        \end{scope}
    \end{tikzpicture}
    \caption{研磨机 UI}
\end{figure}

\begin{itemize}
    \item 左上方: 原材料槽
    \item 左下方: 燃料槽
    \item 右侧:输出槽
\end{itemize}

\subsubsection{压榨机}

压榨机(squeezer) 对原材料压制。

压榨机的合成配方如下:

\begin{figure}[H]
    \centering
    \begin{tikzpicture}
        \draw [recipe grid](0,0) grid (3,3);
        \node (1) at (0.5,2.5) {};
        \node (2) at (1.5,2.5) {};
        \node (3) at (2.5,2.5) {};
        \node (4) at (0.5,1.5)  {};
        \node (5) at (1.5,1.5)  {\includegraphics[width=0.8cm,height=0.8cm]{./images/origin/barrel_side.png}};
        \node (6) at (2.5,1.5)  {};
        \node (7) at (0.5,0.5) {\includegraphics[width=0.8cm,height=0.8cm]{./images/origin/oak_planks.png}};
        \node (8) at (1.5,0.5)  {\includegraphics[width=0.8cm,height=0.8cm]{./images/origin/oak_planks.png}};
        \node (9) at (2.5,0.5) {\includegraphics[width=0.8cm,height=0.8cm]{./images/origin/oak_planks.png}};
        \draw [recipe grid] (6,1) grid (7,2);
        \node (10) at (6.5,1.5) {\includegraphics[width=0.8cm,height=0.8cm]{./images/mod/squeezer_icon.png}};
        \begin{scope}
            \draw [-{Stealth},thick] (3.5,1.5) -- (5.5,1.5);
            \node (text) [font=\small] at (4.5,1.2) {工作台};
        \end{scope}
    \end{tikzpicture}
    \caption{面粉}
\end{figure}

压榨机类似与原版熔炉,但压榨机不需要原材料,对应的压榨时间较长。

\begin{table}[H]
    \centering
    \caption{压榨机属性}
    \setlength{\tabcolsep}{4mm}
    \begin{tabular}{c|cc|c}
        \toprule
        \textbf{属性} & \textbf{说明} & \textbf{属性} & \textbf{说明} \\
        \midrule
        大小 & 1x1.2 & 作用 & 压榨原材料 \\
        饥饿度增量 & - & 经验 & 0.35 \\
        \bottomrule
    \end{tabular}
\end{table}

其熔炉面板应具有以下结构:

\begin{figure}[H]
    \centering
    \begin{tikzpicture}
        \draw [recipe grid](-2,0) grid (1,1);
        \draw [recipe grid](4,0) grid (5,1);
        \begin{scope}
            \draw [-{Stealth},thick] (1.5,0.5) -- (3.5,0.5);
        \end{scope}
    \end{tikzpicture}
    \caption{压榨机 UI}
\end{figure}

\begin{itemize}
    \item 左侧: 原材料槽
    \item 右侧:输出槽
\end{itemize}

\subsection{产品熔炉}

\subsubsection{炸锅}

炸锅 (fryer) 用于油炸食品,炸锅需要放在熔炉上面使用。

炸锅的合成配方如下:

\begin{figure}[H]
    \centering
    \begin{tikzpicture}
        \draw [recipe grid](0,0) grid (3,3);
        \node (1) at (0.5,2.5) {};
        \node (2) at (1.5,2.5) {};
        \node (3) at (2.5,2.5) {};
        \node (4) at (0.5,1.5)  {\includegraphics[width=0.8cm,height=0.8cm]{./images/origin/iron_ingot.png}};
        \node (5) at (1.5,1.5)  {\includegraphics[width=0.8cm,height=0.8cm]{./images/origin/iron_ingot.png}};
        \node (6) at (2.5,1.5)  {\includegraphics[width=0.8cm,height=0.8cm]{./images/origin/iron_ingot.png}};
        \node (7) at (0.5,0.5) {};
        \node (8) at (1.5,0.5)  {\includegraphics[width=0.8cm,height=0.8cm]{./images/origin/iron_ingot.png}};
        \node (9) at (2.5,0.5) {\includegraphics[width=0.8cm,height=0.8cm]{./images/origin/iron_ingot.png}};
        \draw [recipe grid] (6,1) grid (7,2);
        \node (10) at (6.5,1.5) {\includegraphics[width=0.8cm,height=0.8cm]{./images/mod/fryer_icon.png}};
        \begin{scope}
            \draw [-{Stealth},thick] (3.5,1.5) -- (5.5,1.5);
            \node (text) [font=\small] at (4.5,1.2) {工作台};
        \end{scope}
    \end{tikzpicture}
    \caption{面粉}
\end{figure}

\begin{table}[H]
    \centering
    \caption{炸锅属性}
    \setlength{\tabcolsep}{4mm}
    \begin{tabular}{c|cc|c}
        \toprule
        \textbf{属性} & \textbf{说明} & \textbf{属性} & \textbf{说明} \\
        \midrule
        大小 & 1x1 & 作用 & 油炸食物 \\
        饥饿度增量 & 3-7 & 经验 & 0.35 \\
        \bottomrule
    \end{tabular}
\end{table}

其熔炉面板应具有以下结构:

\begin{figure}[H]
    \centering
    \begin{tikzpicture}
        \draw [recipe grid](-4,-1) grid (-3,0);
        \draw [recipe grid](-4,1) grid (-3,2);
        \draw [] (-2,-1) rectangle (-1.5,2);
        \draw [recipe grid](0,1) grid (1,2);
        \draw [recipe grid](4,1) grid (5,2);
        \draw [recipe grid](2,-1) grid (3,0);
        \begin{scope}
            \draw [-{Stealth},thick] (1.5,1.5) -- (3.5,1.5);
        \end{scope}
    \end{tikzpicture}
    \caption{炸锅 UI}
\end{figure}

\begin{itemize}
    \item 左侧: 油量表与进油出油桶
    \item 右侧: 炸锅本体
\end{itemize}

\subsubsection{平底锅}

平底锅 (pan) 用于炒饭。

平底锅的合成配方如下:

\begin{figure}[H]
    \centering
    \begin{tikzpicture}
        \draw [recipe grid](0,0) grid (3,3);
        \node (1) at (0.5,2.5) {};
        \node (2) at (1.5,2.5) {};
        \node (3) at (2.5,2.5) {};
        \node (4) at (0.5,1.5)  {\includegraphics[width=0.8cm,height=0.8cm]{./images/origin/oak_planks.png}};
        \node (5) at (1.5,1.5)  {\includegraphics[width=0.8cm,height=0.8cm]{./images/origin/iron_ingot.png}};
        \node (6) at (2.5,1.5)  {\includegraphics[width=0.8cm,height=0.8cm]{./images/origin/iron_ingot.png}};
        \node (7) at (0.5,0.5) {};
        \node (8) at (1.5,0.5)  {\includegraphics[width=0.8cm,height=0.8cm]{./images/origin/iron_ingot.png}};
        \node (9) at (2.5,0.5) {\includegraphics[width=0.8cm,height=0.8cm]{./images/origin/iron_ingot.png}};
        \draw [recipe grid] (6,1) grid (7,2);
        \node (10) at (6.5,1.5) {\includegraphics[width=0.8cm,height=0.8cm]{./images/mod/pan_icon.png}};
        \begin{scope}
            \draw [-{Stealth},thick] (3.5,1.5) -- (5.5,1.5);
            \node (text) [font=\small] at (4.5,1.2) {工作台};
        \end{scope}
    \end{tikzpicture}
    \caption{平底锅}
\end{figure}

其熔炉面板应具有以下结构:

\begin{figure}[H]
    \centering
    \begin{tikzpicture}
        \begin{scope}[yshift=-0.5cm]
            \draw [recipe grid](-2,1) grid (1,3);
        \end{scope}
        \draw [recipe grid](4,1) grid (5,2);
        \draw [recipe grid](-1,-1) grid (0,-0);
        \draw [recipe grid](2,-1) grid (3,0);
        \draw [recipe grid](6,-1) grid (7,3);
        \begin{scope}
            \draw [-{Stealth},thick] (1.5,1.5) -- (3.5,1.5);
        \end{scope}
    \end{tikzpicture}
    \caption{平底锅 UI}
\end{figure}

\begin{itemize}
    \item 左侧: 油量表与进油出油桶
    \item 右侧: 平底锅本体
\end{itemize}

\subsubsection{烧烤架}

烧烤架 (grill) 用于烧烤食品,烧烤架需要放在熔炉上面使用。

烧烤架的合成配方如下:

\begin{figure}[H]
    \centering
    \begin{tikzpicture}
        \draw [recipe grid](0,0) grid (3,3);
        \node (1) at (0.5,2.5) {};
        \node (2) at (1.5,2.5) {};
        \node (3) at (2.5,2.5) {};
        \node (4) at (0.5,1.5)  {\includegraphics[width=0.8cm,height=0.8cm]{./images/origin/iron_ingot.png}};
        \node (5) at (1.5,1.5)  {\includegraphics[width=0.8cm,height=0.8cm]{./images/origin/iron_ingot.png}};
        \node (6) at (2.5,1.5)  {\includegraphics[width=0.8cm,height=0.8cm]{./images/origin/iron_ingot.png}};
        \node (7) at (0.5,0.5) {\includegraphics[width=0.8cm,height=0.8cm]{./images/origin/iron_ingot.png}};
        \node (8) at (1.5,0.5)  {};
        \node (9) at (2.5,0.5) {\includegraphics[width=0.8cm,height=0.8cm]{./images/origin/iron_ingot.png}};
        \draw [recipe grid] (6,1) grid (7,2);
        \node (10) at (6.5,1.5) {\includegraphics[width=0.8cm,height=0.8cm]{./images/mod/grill_icon.png}};
        \begin{scope}
            \draw [-{Stealth},thick] (3.5,1.5) -- (5.5,1.5);
            \node (text) [font=\small] at (4.5,1.2) {工作台};
        \end{scope}
    \end{tikzpicture}
    \caption{烤架}
\end{figure}

此外,还有一种简易烧烤架(simple\_grill),仅能制作三星及以下食物。

\begin{figure}[H]
    \centering
    \begin{tikzpicture}
        \draw [recipe grid](0,0) grid (3,3);
        \node (1) at (0.5,2.5) {};
        \node (2) at (1.5,2.5) {};
        \node (3) at (2.5,2.5) {};
        \node (4) at (0.5,1.5)  {};
        \node (5) at (1.5,1.5)  {\includegraphics[width=0.8cm,height=0.8cm]{./images/origin/oak_log.png}};
        \node (6) at (2.5,1.5)  {};
        \node (7) at (0.5,0.5) {\includegraphics[width=0.8cm,height=0.8cm]{./images/origin/cobblestone.png}};
        \node (8) at (1.5,0.5)  {\includegraphics[width=0.8cm,height=0.8cm]{./images/origin/oak_log.png}};
        \node (9) at (2.5,0.5) {\includegraphics[width=0.8cm,height=0.8cm]{./images/origin/cobblestone.png}};
        \draw [recipe grid] (6,1) grid (7,2);
        \node (10) at (6.5,1.5) {\includegraphics[width=0.8cm,height=0.8cm]{./images/mod/simple_grill_icon.png}};
        \begin{scope}
            \draw [-{Stealth},thick] (3.5,1.5) -- (5.5,1.5);
            \node (text) [font=\small] at (4.5,1.2) {工作台};
        \end{scope}
    \end{tikzpicture}
    \caption{简易烤架}
\end{figure}


\begin{table}[H]
    \centering
    \caption{烧烤架属性}
    \setlength{\tabcolsep}{4mm}
    \begin{tabular}{c|cc|c}
        \toprule
        \textbf{属性} & \textbf{说明} & \textbf{属性} & \textbf{说明} \\
        \midrule
        大小 & 1x1 & 作用 & 油炸食物 \\
        饥饿度增量 & 3-7 & 经验 & 0.35 \\
        \bottomrule
    \end{tabular}
\end{table}

其熔炉面板应具有以下结构:

\begin{figure}[H]
    \centering
    \begin{tikzpicture}
        \draw [recipe grid](0,1) grid (1,2);
        \draw [recipe grid](4,1) grid (5,2);
        \draw [recipe grid](2,-1) grid (3,0);
        \begin{scope}
            \draw [-{Stealth},thick] (1.5,1.5) -- (3.5,1.5);
        \end{scope}
    \end{tikzpicture}
    \caption{烧烤架 UI}
\end{figure}

\subsubsection{炖锅}

炖锅 (stew\_pot) 用于炖制食品。

炖锅的合成配方如下:

\begin{figure}[H]
    \centering
    \begin{tikzpicture}
        \draw [recipe grid](0,0) grid (3,3);
        \node (1) at (0.5,2.5) {};
        \node (2) at (1.5,2.5) {};
        \node (3) at (2.5,2.5) {};
        \node (4) at (0.5,1.5)  {\includegraphics[width=0.8cm,height=0.8cm]{./images/origin/iron_ingot.png}};
        \node (5) at (1.5,1.5)  {};
        \node (6) at (2.5,1.5)  {\includegraphics[width=0.8cm,height=0.8cm]{./images/origin/iron_ingot.png}};
        \node (7) at (0.5,0.5) {\includegraphics[width=0.8cm,height=0.8cm]{./images/origin/iron_ingot.png}};
        \node (8) at (1.5,0.5)  {\includegraphics[width=0.8cm,height=0.8cm]{./images/origin/iron_ingot.png}};
        \node (9) at (2.5,0.5) {\includegraphics[width=0.8cm,height=0.8cm]{./images/origin/iron_ingot.png}};
        \draw [recipe grid] (6,1) grid (7,2);
        \node (10) at (6.5,1.5) {\includegraphics[width=0.8cm,height=0.8cm]{./images/mod/stew_pot_icon.png}};
        \begin{scope}
            \draw [-{Stealth},thick] (3.5,1.5) -- (5.5,1.5);
            \node (text) [font=\small] at (4.5,1.2) {工作台};
        \end{scope}
    \end{tikzpicture}
    \caption{烤架}
\end{figure}

此外,还有一种简易炖锅(simple\_stew\_pot),仅能制作三星及以下食物。

\begin{figure}[H]
    \centering
    \begin{tikzpicture}
        \draw [recipe grid](0,0) grid (3,3);

        \node (1) at (0.5,2.5) {};
        \node (2) at (1.5,2.5) {};
        \node (3) at (2.5,2.5) {};
        \node (4) at (0.5,1.5)  {};
        \node (5) at (1.5,1.5)  {\includegraphics[width=0.8cm,height=0.8cm]{./images/mod/stew_pot_icon.png}};
        \node (6) at (2.5,1.5)  {};
        \node (7) at (0.5,0.5) {};
        \node (8) at (1.5,0.5)  {\includegraphics[width=0.8cm,height=0.8cm]{./images/origin/oak_log.png}};
        \node (9) at (2.5,0.5) {};
        \draw [recipe grid] (6,1) grid (7,2);
        \node (10) at (6.5,1.5) {\includegraphics[width=0.8cm,height=0.8cm]{./images/mod/simple_stew_pot_icon.png}};
        \begin{scope}
            \draw [-{Stealth},thick] (3.5,1.5) -- (5.5,1.5);
            \node (text) [font=\small] at (4.5,1.2) {工作台};
        \end{scope}
    \end{tikzpicture}
    \caption{简易烤架}
\end{figure}


\begin{table}[H]
    \centering
    \caption{炖锅属性}
    \setlength{\tabcolsep}{4mm}
    \begin{tabular}{c|cc|c}
        \toprule
        \textbf{属性} & \textbf{说明} & \textbf{属性} & \textbf{说明} \\
        \midrule
        大小 & 1x1 & 作用 & 油炸食物 \\
        饥饿度增量 & 1-2 & 经验 & 0.35 \\
        \bottomrule
    \end{tabular}
\end{table}

其熔炉面板应具有以下结构:

\begin{figure}[H]
    \centering
    \begin{tikzpicture}
        \begin{scope}[yshift=0.5cm]
            \draw [recipe grid](-5,-1) grid (-4,0);
            \draw [recipe grid](-5,1) grid (-4,2);
            \draw [] (-3,-1) rectangle (-2.5,2);
        \end{scope}
        \begin{scope}[yshift=-0.5cm]
            \draw [recipe grid](-2,1) grid (1,3);
        \end{scope}
        \draw [recipe grid](4,1) grid (5,2);
        \draw [recipe grid](-1,-1) grid (0,-0);
        \draw [recipe grid](2,-1) grid (3,0);
        \begin{scope}
            \draw [-{Stealth},thick] (1.5,1.5) -- (3.5,1.5);
        \end{scope}
    \end{tikzpicture}
    \caption{炖锅 UI}
\end{figure}

炖锅的 UI 比较复杂,其中左侧为水表,中间为本体,右下角为汤碗槽。

\subsubsection{蒸笼}

蒸笼 (food\_steamer) 用于烧烤食品,蒸笼需要放在熔炉上面使用。

蒸笼的合成配方如下:

\begin{figure}[H]
    \centering
    \begin{tikzpicture}
        \draw [recipe grid](0,0) grid (3,3);
        \node (1) at (0.5,2.5) {\includegraphics[width=0.8cm,height=0.8cm]{./images/origin/oak_planks.png}};
        \node (2) at (1.5,2.5) {};
        \node (3) at (2.5,2.5) {\includegraphics[width=0.8cm,height=0.8cm]{./images/origin/oak_planks.png}};
        \node (4) at (0.5,1.5)  {\includegraphics[width=0.8cm,height=0.8cm]{./images/origin/oak_planks.png}};
        \node (5) at (1.5,1.5)  {};
        \node (6) at (2.5,1.5)  {\includegraphics[width=0.8cm,height=0.8cm]{./images/origin/oak_planks.png}};
        \node (7) at (0.5,0.5) {\includegraphics[width=0.8cm,height=0.8cm]{./images/origin/oak_planks.png}};
        \node (8) at (1.5,0.5)  {};
        \node (9) at (2.5,0.5) {\includegraphics[width=0.8cm,height=0.8cm]{./images/origin/oak_planks.png}};
        \draw [recipe grid] (6,1) grid (7,2);
        \node (10) at (6.5,1.5) {\includegraphics[width=0.8cm,height=0.8cm]{./images/mod/food_steamer_icon.png}};
        \begin{scope}
            \draw [-{Stealth},thick] (3.5,1.5) -- (5.5,1.5);
            \node (text) [font=\small] at (4.5,1.2) {工作台};
        \end{scope}
    \end{tikzpicture}
    \caption{烤架}
\end{figure}

\begin{table}[H]
    \centering
    \caption{蒸笼属性}
    \setlength{\tabcolsep}{4mm}
    \begin{tabular}{c|cc|c}
        \toprule
        \textbf{属性} & \textbf{说明} & \textbf{属性} & \textbf{说明} \\
        \midrule
        大小 & 1x1 & 作用 & 蒸制食物 \\
        饥饿度增量 & 0 & 经验 & 0.35 \\
        \bottomrule
    \end{tabular}
\end{table}

其熔炉面板应具有以下结构:

\begin{figure}[H]
    \centering
    \begin{tikzpicture}
        \draw [recipe grid](-4,-1) grid (-3,0);
        \draw [recipe grid](-4,1) grid (-3,2);
        \draw [] (-2,-1) rectangle (-1.5,2);
        \draw [recipe grid](0,1) grid (1,2);
        \draw [recipe grid](4,1) grid (5,2);
        \draw [recipe grid](2,-1) grid (3,0);
        \begin{scope}
            \draw [-{Stealth},thick] (1.5,1.5) -- (3.5,1.5);
        \end{scope}
    \end{tikzpicture}
    \caption{蒸笼 UI}
\end{figure}

\subsection{响应方块}

\subsubsection{晾肉架}

晾肉架(fence) 可以制成烧烤类食物。

晾肉架的合成配方如下:

\begin{figure}[H]
    \centering
    \begin{tikzpicture}
        \draw [recipe grid](0,0) grid (3,3);
        \node (1) at (0.5,2.5) {};
        \node (2) at (1.5,2.5) {};
        \node (3) at (2.5,2.5) {};
        \node (4) at (0.5,1.5)  {\includegraphics[width=0.8cm,height=0.8cm]{./images/origin/oak_planks.png}};
        \node (5) at (1.5,1.5)  {\includegraphics[width=0.8cm,height=0.8cm]{./images/origin/oak_planks.png}};
        \node (6) at (2.5,1.5)  {\includegraphics[width=0.8cm,height=0.8cm]{./images/origin/oak_planks.png}};
        \node (7) at (0.5,0.5) {\includegraphics[width=0.8cm,height=0.8cm]{./images/origin/oak_planks.png}};
        \node (8) at (1.5,0.5)  {\includegraphics[width=0.8cm,height=0.8cm]{./images/origin/string.png}};
        \node (9) at (2.5,0.5) {\includegraphics[width=0.8cm,height=0.8cm]{./images/origin/oak_planks.png}};
        \draw [recipe grid] (6,1) grid (7,2);
        \node (10) at (6.5,1.5) {\includegraphics[width=0.8cm,height=0.8cm]{./images/mod/meal_shelf_icon.png}};
        \begin{scope}
            \draw [-{Stealth},thick] (3.5,1.5) -- (5.5,1.5);
            \node (text) [font=\small] at (4.5,1.2) {工作台};
        \end{scope}
    \end{tikzpicture}
    \caption{面粉}
\end{figure}

晾肉架不通过 UI 合成物品,而是通过交互获得新物品。

\begin{table}[H]
    \centering
    \caption{晾肉架属性}
    \setlength{\tabcolsep}{4mm}
    \begin{tabular}{c|cc|c}
        \toprule
        \textbf{属性} & \textbf{说明} & \textbf{属性} & \textbf{说明} \\
        \midrule
        大小 & 1x1 & 作用 & 制作烧烤 \\
        饥饿度增量 & 2-5 & 经验 & 0.35 \\
        \bottomrule
    \end{tabular}
\end{table}

晾肉架需要以下两种物品:
\begin{itemize}
    \item 燃料: 使用后晾肉架开启燃烧状态。
    \item 食物: 可以在晾肉架上烤制。
\end{itemize}

\newpage